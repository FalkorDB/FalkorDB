\newpage
%%%%%%%%%%%%%%%%%%%%%%%%%%%%%%%%%%%%%%%%%%%%%%%%%%%%%%%%%%%%%%%%%%%%%%%%%%%%%%%%
\section{Printing GraphBLAS objects} %%%%%%%%%%%%%%%%%%%%%%%%%%%%%%%%%%%%%%%%%%%
%%%%%%%%%%%%%%%%%%%%%%%%%%%%%%%%%%%%%%%%%%%%%%%%%%%%%%%%%%%%%%%%%%%%%%%%%%%%%%%%
\label{fprint}

The ten different objects handled by SuiteSparse:GraphBLAS are all opaque,
although nearly all of their contents can be extracted via methods such as
\verb'GrB_Matrix_extractTuples', \verb'GrB_Matrix_extractElement',
\verb'GrB_get', and so on.  The GraphBLAS C API has no mechanism for
printing all the contents of GraphBLAS objects, but this is helpful for
debugging.  Ten type-specific methods and two type-generic methods are
provided:

\vspace{0.2in}
{\footnotesize
\begin{tabular}{ll}
\hline
\verb'GxB_Type_fprint'         & print and check a \verb'GrB_Type' \\
\verb'GxB_UnaryOp_fprint'      & print and check a \verb'GrB_UnaryOp' \\
\verb'GxB_BinaryOp_fprint'     & print and check a \verb'GrB_BinaryOp' \\
\verb'GxB_IndexUnaryOp_fprint' & print and check a \verb'GrB_IndexUnaryOp' \\
\verb'GxB_IndexBinaryOp_fprint'& print and check a \verb'GxB_IndexBinaryOp' \\
\verb'GxB_Monoid_fprint'       & print and check a \verb'GrB_Monoid' \\
\verb'GxB_Semiring_fprint'     & print and check a \verb'GrB_Semiring' \\
\verb'GxB_Descriptor_fprint'   & print and check a \verb'GrB_Descriptor' \\
\verb'GxB_Context_fprint'      & print and check a \verb'GxB_Context' \\
\verb'GxB_Matrix_fprint'       & print and check a \verb'GrB_Matrix' \\
\verb'GxB_Vector_fprint'       & print and check a \verb'GrB_Vector' \\
\verb'GxB_Scalar_fprint'       & print and check a \verb'GrB_Scalar' \\
\hline
\verb'GxB_fprint'             & print/check any object to a file \\
\verb'GxB_print'              & print/check any object to \verb'stdout' \\
\hline
\end{tabular}
}
\vspace{0.2in}

These methods do not modify the status of any object, and thus they
cannot return an error string for use by \verb'GrB_error'.

If a matrix or vector
has not been completed, the pending computations are guaranteed to {\em not} be
performed. The reason is simple.  It is possible for a bug in the user
application (such as accessing memory outside the bounds of an array) to mangle
the internal content of a GraphBLAS object, and the \verb'GxB_*print' methods
can be helpful tools to track down this bug.  If \verb'GxB_*print' attempted to
complete any computations prior to printing or checking the contents of the
matrix or vector, then further errors could occur, including a segfault.

By contrast, GraphBLAS methods and operations that return values into
user-provided arrays or variables might finish pending operations before the
return these values, and this would change their state.  Since they do not
change the state of any object, the \verb'GxB_*print' methods provide a useful
method for debugging, and for a quick understanding of what GraphBLAS is
computing while developing a user application.

Each of the methods has a parameter of type \verb'GxB_Print_Level' that
specifies the amount to print:

{\footnotesize
\begin{verbatim}
typedef enum
{
    GxB_SILENT = 0,     // nothing is printed, just check the object
    GxB_SUMMARY = 1,    // print a terse summary
    GxB_SHORT = 2,      // short description, about 30 entries of a matrix
    GxB_COMPLETE = 3,   // print the entire contents of the object
    GxB_SHORT_VERBOSE = 4,    // GxB_SHORT but with "%.15g" for doubles
    GxB_COMPLETE_VERBOSE = 5  // GxB_COMPLETE but with "%.15g" for doubles
}
GxB_Print_Level ; \end{verbatim}}

The ten type-specific functions include an additional argument, the
\verb'name' string.  The \verb'name' is printed at the beginning of the display
(assuming the print level is not \verb'GxB_SILENT') so that the object can be
more easily identified in the output.  For the type-generic methods
\verb'GxB_fprint' and \verb'GxB_print', the \verb'name' string is the variable
name of the object itself.

If the file \verb'f' is \verb'NULL', \verb'stdout' is used.
If \verb'name' is \verb'NULL', it is treated
as the empty string.  These are not error conditions.

The methods check their input objects carefully and extensively, even when
\verb'pr' is equal to \verb'GxB_SILENT'.  The following error codes can be
returned:

\begin{packed_itemize}
\item \verb'GrB_SUCCESS':               object is valid
\item \verb'GrB_UNINITIALIZED_OBJECT':  object is not initialized
\item \verb'GrB_INVALID_OBJECT':        object is not valid
\item \verb'GrB_NULL_POINTER':          object is a NULL pointer
\item \verb'GrB_INVALID_VALUE':         \verb'fprintf' returned an I/O error.
\end{packed_itemize}

The content of any GraphBLAS object is opaque, and subject to change.  As a
result, the exact content and format of what is printed is
implementation-dependent, and will change from version to version of
SuiteSparse:GraphBLAS.  Do not attempt to rely on the exact content or format
by trying to parse the resulting output via another program.  The intent of
these functions is to produce a report of an object for visual inspection.  If
the user application needs to extract content from a GraphBLAS matrix or
vector, use \verb'GrB_*_extractTuples' or the import/export methods instead.

GraphBLAS matrices and vectors are zero-based, where indices of an $n$-by-$n$
matrix are in the range 0 to $n-1$.  However, MATLAB, Octave, and Julia prefer
to print their matrices and vectors as one-based.  To enable 1-based printing,
use \verb'GrB_set (GrB_GLOBAL, true, GxB_PRINT_1BASED)'.  Printing is done as zero-based by
default.

% \newpage
%===============================================================================
\subsection{{\sf GxB\_fprint:} Print a GraphBLAS object to a file} %============
%===============================================================================

\begin{mdframed}[userdefinedwidth=6in]
{\footnotesize
\begin{verbatim}
GrB_Info GxB_fprint                 // print and check a GraphBLAS object
(
    GrB_<objecttype> object,        // object to print and check
    int pr,                         // print level (GxB_Print_Level)
    FILE *f                         // file for output
) ;
\end{verbatim} } \end{mdframed}

The \verb'GxB_fprint' function prints the contents of any of the ten GraphBLAS
objects to the file \verb'f'.  If \verb'f' is \verb'NULL', the results are
printed to \verb'stdout'.  For example, to print the entire contents of a
matrix \verb'A' to the file \verb'f', use
\verb'GxB_fprint (A, GxB_COMPLETE, f)'.

%===============================================================================
\subsection{{\sf GxB\_print:} Print a GraphBLAS object to {\sf stdout}} %=======
%===============================================================================
\label{gxb_print}

\begin{mdframed}[userdefinedwidth=6in]
{\footnotesize
\begin{verbatim}
GrB_Info GxB_print                  // print and check a GrB_Vector
(
    GrB_<objecttype> object,        // object to print and check
    int pr,                         // print level (GxB_Print_Level)
) ;
\end{verbatim} } \end{mdframed}

\verb'GxB_print' is the same as \verb'GxB_fprint', except that it prints the
contents of the object to \verb'stdout' instead of a file \verb'f'.  For
example, to print the entire contents of a matrix \verb'A',  use
\verb'GxB_print (A, GxB_COMPLETE)'.

%===============================================================================
\subsection{{\sf GxB\_Type\_fprint:} Print a {\sf GrB\_Type}}
%===============================================================================

\begin{mdframed}[userdefinedwidth=6in]
{\footnotesize
\begin{verbatim}
GrB_Info GxB_Type_fprint            // print and check a GrB_Type
(
    GrB_Type type,                  // object to print and check
    const char *name,               // name of the object
    int pr,                         // print level (GxB_Print_Level)
    FILE *f                         // file for output
) ;
\end{verbatim} } \end{mdframed}

For example, \verb'GxB_Type_fprint (GrB_BOOL, "boolean type", GxB_COMPLETE, f)'
prints the contents of the \verb'GrB_BOOL' object to the file \verb'f'.

\newpage
%===============================================================================
\subsection{{\sf GxB\_UnaryOp\_fprint:} Print a {\sf GrB\_UnaryOp}}
%===============================================================================

\begin{mdframed}[userdefinedwidth=6in]
{\footnotesize
\begin{verbatim}
GrB_Info GxB_UnaryOp_fprint         // print and check a GrB_UnaryOp
(
    GrB_UnaryOp unaryop,            // object to print and check
    const char *name,               // name of the object
    int pr,                         // print level (GxB_Print_Level)
    FILE *f                         // file for output
) ;
\end{verbatim} } \end{mdframed}

For example,
\verb'GxB_UnaryOp_fprint (GrB_LNOT, "not", GxB_COMPLETE, f)'
prints the \verb'GrB_LNOT' unary operator to the file \verb'f'.


%===============================================================================
\subsection{{\sf GxB\_BinaryOp\_fprint:} Print a {\sf GrB\_BinaryOp}}
%===============================================================================

\begin{mdframed}[userdefinedwidth=6in]
{\footnotesize
\begin{verbatim}
GrB_Info GxB_BinaryOp_fprint        // print and check a GrB_BinaryOp
(
    GrB_BinaryOp binaryop,          // object to print and check
    const char *name,               // name of the object
    int pr,                         // print level (GxB_Print_Level)
    FILE *f                         // file for output
) ;
\end{verbatim} } \end{mdframed}

For example,
\verb'GxB_BinaryOp_fprint (GrB_PLUS_FP64, "plus", GxB_COMPLETE, f)' prints the
\verb'GrB_PLUS_FP64' binary operator to the file \verb'f'.


%===============================================================================
\subsection{{\sf GxB\_IndexUnaryOp\_fprint:} Print a {\sf GrB\_IndexUnaryOp}}
%===============================================================================

\begin{mdframed}[userdefinedwidth=6in]
{\footnotesize
\begin{verbatim}
GrB_Info GxB_IndexUnaryOp_fprint    // print and check a GrB_IndexUnaryOp
(
    GrB_IndexUnaryOp op,            // object to print and check
    const char *name,               // name of the object
    int pr,                         // print level (GxB_Print_Level)
    FILE *f                         // file for output
) ;
\end{verbatim} } \end{mdframed}

For example,
\verb'GrB_IndexUnaryOp_fprint (GrB_TRIL, "tril", GxB_COMPLETE, f)' prints
the \verb'GrB_TRIL' index-unary operator to the file \verb'f'.

\newpage
%===============================================================================
\subsection{{\sf GxB\_IndexBinaryOp\_fprint:} Print a {\sf GxB\_IndexBinaryOp}}
%===============================================================================

\begin{mdframed}[userdefinedwidth=6in]
{\footnotesize
\begin{verbatim}
GrB_Info GxB_IndexBinaryOp_fprint   // print and check a GxB_IndexBinaryOp
(
    GxB_IndexBinaryOp op,           // object to print and check
    const char *name,               // name of the object
    int pr,                         // print level (GxB_Print_Level)
    FILE *f                         // file for output
) ;
\end{verbatim} } \end{mdframed}

%===============================================================================
\subsection{{\sf GxB\_Monoid\_fprint:} Print a {\sf GrB\_Monoid}}
%===============================================================================

\begin{mdframed}[userdefinedwidth=6in]
{\footnotesize
\begin{verbatim}
GrB_Info GxB_Monoid_fprint          // print and check a GrB_Monoid
(
    GrB_Monoid monoid,              // object to print and check
    const char *name,               // name of the object
    int pr,                         // print level (GxB_Print_Level)
    FILE *f                         // file for output
) ;
\end{verbatim} } \end{mdframed}

For example,
\verb'GxB_Monoid_fprint (GxB_PLUS_FP64_MONOID, "plus monoid",'
\verb'GxB_COMPLETE, f)'
prints the predefined \verb'GxB_PLUS_FP64_MONOID' (based on the binary
operator \verb'GrB_PLUS_FP64') to the file \verb'f'.

%===============================================================================
\subsection{{\sf GxB\_Semiring\_fprint:} Print a {\sf GrB\_Semiring}}
%===============================================================================

\begin{mdframed}[userdefinedwidth=6in]
{\footnotesize
\begin{verbatim}
GrB_Info GxB_Semiring_fprint        // print and check a GrB_Semiring
(
    GrB_Semiring semiring,          // object to print and check
    const char *name,               // name of the object
    int pr,                         // print level (GxB_Print_Level)
    FILE *f                         // file for output
) ;
\end{verbatim} } \end{mdframed}

For example,
\verb'GxB_Semiring_fprint (GxB_PLUS_TIMES_FP64, "standard",'
\verb'GxB_COMPLETE, f)'
prints the predefined \verb'GxB_PLUS_TIMES_FP64' semiring to the file \verb'f'.

\newpage
%===============================================================================
\subsection{{\sf GxB\_Descriptor\_fprint:} Print a {\sf GrB\_Descriptor}}
%===============================================================================

\begin{mdframed}[userdefinedwidth=6in]
{\footnotesize
\begin{verbatim}
GrB_Info GxB_Descriptor_fprint      // print and check a GrB_Descriptor
(
    GrB_Descriptor descriptor,      // object to print and check
    const char *name,               // name of the object
    int pr,                         // print level (GxB_Print_Level)
    FILE *f                         // file for output
) ;
\end{verbatim} } \end{mdframed}

For example,
\verb'GxB_Descriptor_fprint (d, "descriptor", GxB_COMPLETE, f)'
prints the descriptor \verb'd' to the file \verb'f'.

%===============================================================================
\subsection{{\sf GxB\_Context\_fprint:} Print a {\sf GxB\_Context}}
%===============================================================================
\label{context_print}

\begin{mdframed}[userdefinedwidth=6in]
{\footnotesize
\begin{verbatim}
GrB_Info GxB_Context_fprint         // print and check a GxB_Context
(
    GxB_Context Context,            // object to print and check
    const char *name,               // name of the object
    int pr,                         // print level (GxB_Print_Level)
    FILE *f                         // file for output
) ;
\end{verbatim} } \end{mdframed}

This method can be used to print the context created for a user thread,
or the contents of the \verb'GxB_CONTEXT_WORLD' object.

%===============================================================================
\subsection{{\sf GxB\_Matrix\_fprint:} Print a {\sf GrB\_Matrix}}
%===============================================================================

\begin{mdframed}[userdefinedwidth=6in]
{\footnotesize
\begin{verbatim}
GrB_Info GxB_Matrix_fprint          // print and check a GrB_Matrix
(
    GrB_Matrix A,                   // object to print and check
    const char *name,               // name of the object
    int pr,                         // print level (GxB_Print_Level)
    FILE *f                         // file for output
) ;
\end{verbatim} } \end{mdframed}

For example, \verb'GxB_Matrix_fprint (A, "my matrix", GxB_SHORT, f)'
prints about 30 entries from the matrix \verb'A' to the file \verb'f'.


\newpage
%===============================================================================
\subsection{{\sf GxB\_Vector\_fprint:} Print a {\sf GrB\_Vector}}
%===============================================================================

\begin{mdframed}[userdefinedwidth=6in]
{\footnotesize
\begin{verbatim}
GrB_Info GxB_Vector_fprint          // print and check a GrB_Vector
(
    GrB_Vector v,                   // object to print and check
    const char *name,               // name of the object
    int pr,                         // print level (GxB_Print_Level)
    FILE *f                         // file for output
) ;
\end{verbatim} } \end{mdframed}

For example, \verb'GxB_Vector_fprint (v, "my vector", GxB_SHORT, f)'
prints about 30 entries from the vector \verb'v' to the file \verb'f'.

%===============================================================================
\subsection{{\sf GxB\_Scalar\_fprint:} Print a {\sf GrB\_Scalar}}
%===============================================================================

\begin{mdframed}[userdefinedwidth=6in]
{\footnotesize
\begin{verbatim}
GrB_Info GxB_Scalar_fprint          // print and check a GrB_Scalar
(
    GrB_Scalar s,                   // object to print and check
    const char *name,               // name of the object
    int pr,                         // print level (GxB_Print_Level)
    FILE *f                         // file for output
) ;
\end{verbatim} } \end{mdframed}

For example, \verb'GxB_Scalar_fprint (s, "my scalar", GxB_SHORT, f)'
prints a short description of the scalar \verb's' to the file \verb'f'.

%===============================================================================
\subsection{Performance and portability considerations}
%===============================================================================

Even when the print level is \verb'GxB_SILENT', these methods extensively check
the contents of the objects passed to them, which can take some time.  They
should be considered debugging tools only, not for final use in production.

The return value of the \verb'GxB_*print' methods can be relied upon, but the
output to the file (or \verb'stdout') can change from version to version.  If
these methods are eventually added to the GraphBLAS C API Specification, a
conforming implementation might never print anything at all, regardless of the
\verb'pr' value.  This may be essential if the GraphBLAS library is installed
in a dedicated device, with no file output, for example.

Some implementations may wish to print nothing at all if the matrix is not yet
completed, or just an indication that the matrix has pending operations and
cannot be printed, when non-blocking mode is employed.  In this case, use
\verb'GrB_Matrix_wait', \verb'GrB_Vector_wait', or \verb'GxB_Scalar_wait' to
finish all pending computations first.  If a matrix or vector has pending
operations, SuiteSparse:GraphBLAS prints a list of the {\em pending tuples},
which are the entries not yet inserted into the primary data structure.  It can
also print out entries that remain in the data structure but are awaiting
deletion; these are called {\em zombies} in the output report.

Most of the rest of the report is self-explanatory.

