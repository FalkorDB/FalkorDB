
\newpage
%===============================================================================
\subsection{GraphBLAS vectors: {\sf GrB\_Vector}} %=============================
%===============================================================================
\label{vector}

This section describes a set of methods that create, modify, query,
and destroy a GraphBLAS sparse vector, \verb'GrB_Vector':

\vspace{0.2in}
\noindent
{\footnotesize
\begin{tabular}{lll}
\hline
GraphBLAS function   & purpose                                      & Section \\
\hline
\verb'GrB_Vector_new'            & create a vector                  & \ref{vector_new} \\
\verb'GrB_Vector_wait'           & wait for a vector                & \ref{vector_wait} \\
\verb'GrB_Vector_dup'            & copy a vector                    & \ref{vector_dup} \\
\verb'GrB_Vector_clear'          & clear a vector of all entries    & \ref{vector_clear} \\
\verb'GrB_Vector_size'           & size of a vector                 & \ref{vector_size} \\
\verb'GrB_Vector_nvals'          & number of entries in a vector    & \ref{vector_nvals} \\
\verb'GrB_get'  & get properties of a vector       & \ref{get_set_vector} \\
\verb'GrB_set'  & set properties of a vector       & \ref{get_set_vector} \\
\verb'GrB_Vector_build'          & build a vector from tuples       & \ref{vector_build} \\
\verb'GxB_Vector_build_Scalar'   & build a vector from tuples       & \ref{vector_build_Scalar} \\
\verb'GxB_Vector_build_Vector'          & build a vector from tuples       & \ref{vector_build_Vector} \\
\verb'GxB_Vector_build_Scalar_Vector'   & build a vector from tuples       & \ref{vector_build_Scalar_Vector} \\
\verb'GrB_Vector_setElement'     & add an entry to a vector         & \ref{vector_setElement} \\
\verb'GrB_Vector_extractElement' & get an entry from a vector       & \ref{vector_extractElement} \\
\verb'GxB_Vector_isStoredElement'& check if entry present in vector & \ref{vector_isStoredElement} \\
\verb'GrB_Vector_removeElement'  & remove an entry from a vector    & \ref{vector_removeElement} \\
\verb'GrB_Vector_extractTuples'  & get all entries from a vector    & \ref{vector_extractTuples} \\
\verb'GxB_Vector_extractTuples_Vector'  & get all entries from a vector    & \ref{vector_extractTuples_Vector} \\
\verb'GrB_Vector_resize'         & resize a vector                  & \ref{vector_resize} \\
\verb'GxB_Vector_diag'           & extract a diagonal from a matrix & \ref{vector_diag} \\
\verb'GxB_Vector_memoryUsage'    & memory used by a vector          & \ref{vector_memusage} \\
\verb'GxB_Vector_type'           & type of the matrix               & \ref{vector_type} \\
\verb'GrB_Vector_free'           & free a vector                    & \ref{vector_free} \\
\hline
\hline
% NOTE: GrB_Vector_serialize / deserialize does not appear in the 2.0 C API.
% \verb'GrB_Vector_serializeSize'  & return size of serialized vector & \ref{vector_serialize_size} \\
% \verb'GrB_Vector_serialize'      & serialize a vector               & \ref{vector_serialize} \\
\verb'GxB_Vector_serialize'      & serialize a vector               & \ref{vector_serialize_GxB} \\
% \verb'GrB_Vector_deserialize'    & deserialize a vector             & \ref{vector_deserialize} \\
\verb'GxB_Vector_deserialize'    & deserialize a vector             & \ref{vector_deserialize_GxB} \\
\hline
\hline
\verb'GxB_Vector_sort'          & sort a vector & \ref{vector_sort} \\
\end{tabular}
}

\vspace{0.2in}
Refer to
Section~\ref{serialize_deserialize} for serialization/deserialization methods
and to
Section~\ref{sorting_methods} for sorting methods.

\newpage
%-------------------------------------------------------------------------------
\subsubsection{{\sf GrB\_Vector\_new:}           create a vector}
%-------------------------------------------------------------------------------
\label{vector_new}

\begin{mdframed}[userdefinedwidth=6in]
{\footnotesize
\begin{verbatim}
GrB_Info GrB_Vector_new     // create a new vector with no entries
(
    GrB_Vector *v,          // handle of vector to create
    GrB_Type type,          // type of vector to create
    GrB_Index n             // vector dimension is n-by-1
) ;
\end{verbatim}
} \end{mdframed}

\verb'GrB_Vector_new' creates a new \verb'n'-by-\verb'1' sparse vector with no
entries in it, of the given type.  This is analogous to MATLAB/Octave statement
\verb'v = sparse (n,1)', except that GraphBLAS can create sparse vectors any
type.  The pattern of the new vector is empty.

\begin{alert}
{\bf SPEC:} \verb'n' may be zero, as an extension to the specification.
\end{alert}

%-------------------------------------------------------------------------------
\subsubsection{{\sf GrB\_Vector\_wait:} wait for a vector}
%-------------------------------------------------------------------------------
\label{vector_wait}

\begin{mdframed}[userdefinedwidth=6in]
{\footnotesize
\begin{verbatim}
GrB_Info GrB_wait               // wait for a vector
(
    GrB_Vector w,               // vector to wait for
    int mode                    // GrB_COMPLETE or GrB_MATERIALIZE
) ;
\end{verbatim}
}\end{mdframed}

In non-blocking mode, the computations for a \verb'GrB_Vector' may be delayed.
In this case, the vector is not yet safe to use by multiple independent user
threads.  A user application may force completion of a vector \verb'w' via
\verb'GrB_Vector_wait(&w)' (in v5.2.0), or
\verb'GrB_Vector_wait(w,mode)' (in v6.0.0).
With a \verb'mode' of \verb'GrB_MATERIALIZE',
all pending computations are finished, and different user threads may
simultaneously call GraphBLAS operations that use the vector \verb'w' as an
input parameter.
See Section~\ref{omp_parallelism}
if GraphBLAS is compiled without OpenMP.

\newpage
%-------------------------------------------------------------------------------
\subsubsection{{\sf GrB\_Vector\_dup:}           copy a vector}
%-------------------------------------------------------------------------------
\label{vector_dup}

\begin{mdframed}[userdefinedwidth=6in]
{\footnotesize
\begin{verbatim}
GrB_Info GrB_Vector_dup     // make an exact copy of a vector
(
    GrB_Vector *w,          // handle of output vector to create
    const GrB_Vector u      // input vector to copy
) ;
\end{verbatim}
} \end{mdframed}

\verb'GrB_Vector_dup' makes a deep copy of a sparse vector.
In GraphBLAS, it is possible, and valid, to write the following:

    {\footnotesize
    \begin{verbatim}
    GrB_Vector u, w ;
    GrB_Vector_new (&u, GrB_FP64, n) ;
    w = u ;                         // w is a shallow copy of u  \end{verbatim}}

Then \verb'w' and \verb'u' can be used interchangeably.  However, only a pointer
reference is made, and modifying one of them modifies both, and freeing one of
them leaves the other as a dangling handle that should not be used.
If two different vectors are needed, then this should be used instead:

    {\footnotesize
    \begin{verbatim}
    GrB_Vector u, w ;
    GrB_Vector_new (&u, GrB_FP64, n) ;
    GrB_Vector_dup (&w, u) ;        // like w = u, but making a deep copy \end{verbatim}}

Then \verb'w' and \verb'u' are two different vectors that currently have the
same set of values, but they do not depend on each other.  Modifying one has
no effect on the other.
The \verb'GrB_NAME' is copied into the new vector.

%-------------------------------------------------------------------------------
\subsubsection{{\sf GrB\_Vector\_clear:}         clear a vector of all entries}
%-------------------------------------------------------------------------------
\label{vector_clear}

\begin{mdframed}[userdefinedwidth=6in]
{\footnotesize
\begin{verbatim}
GrB_Info GrB_Vector_clear   // clear a vector of all entries;
(                           // type and dimension remain unchanged.
    GrB_Vector v            // vector to clear
) ;
\end{verbatim}
} \end{mdframed}

\verb'GrB_Vector_clear' clears all entries from a vector.  All values
\verb'v(i)' are now equal to the implicit value, depending on what semiring
ring is used to perform computations on the vector.  The pattern of \verb'v' is
empty, just as if it were created fresh with \verb'GrB_Vector_new'.  Analogous
with \verb'v (:) = sparse(0)' in MATLAB.  The type and dimension of \verb'v' do
not change.  Any pending updates to the vector are discarded.

\newpage
%-------------------------------------------------------------------------------
\subsubsection{{\sf GrB\_Vector\_size:}          return the size of a vector}
%-------------------------------------------------------------------------------
\label{vector_size}

\begin{mdframed}[userdefinedwidth=6in]
{\footnotesize
\begin{verbatim}
GrB_Info GrB_Vector_size    // get the dimension of a vector
(
    GrB_Index *n,           // vector dimension is n-by-1
    const GrB_Vector v      // vector to query
) ;
\end{verbatim}
} \end{mdframed}

\verb'GrB_Vector_size' returns the size of a vector (the number of rows).
Analogous to \verb'n = length(v)' or \verb'n = size(v,1)' in MATLAB.

%-------------------------------------------------------------------------------
\subsubsection{{\sf GrB\_Vector\_nvals:}         return the number of entries in a vector}
%-------------------------------------------------------------------------------
\label{vector_nvals}

\begin{mdframed}[userdefinedwidth=6in]
{\footnotesize
\begin{verbatim}
GrB_Info GrB_Vector_nvals   // get the number of entries in a vector
(
    GrB_Index *nvals,       // vector has nvals entries
    const GrB_Vector v      // vector to query
) ;
\end{verbatim}
} \end{mdframed}

\verb'GrB_Vector_nvals' returns the number of entries in a vector.  Roughly
analogous to \verb'nvals = nnz(v)' in MATLAB, except that the implicit value in
GraphBLAS need not be zero and \verb'nnz' (short for ``number of nonzeros'') in
MATLAB is better described as ``number of entries'' in GraphBLAS.

%-------------------------------------------------------------------------------
\subsubsection{{\sf GrB\_Vector\_build:}         build a vector from a set of tuples}
%-------------------------------------------------------------------------------
\label{vector_build}

\begin{mdframed}[userdefinedwidth=6in]
{\footnotesize
\begin{verbatim}
GrB_Info GrB_Vector_build           // build a vector from (I,X) tuples
(
    GrB_Vector w,                   // vector to build
    const GrB_Index *I,             // array of row indices of tuples
    const <type> *X,                // array of values of tuples
    GrB_Index nvals,                // number of tuples
    const GrB_BinaryOp dup          // binary function to assemble duplicates
) ;
\end{verbatim}
} \end{mdframed}

\verb'GrB_Vector_build' constructs a sparse vector \verb'w' from a set of
tuples, \verb'I' and \verb'X', each of length \verb'nvals'.  The vector
\verb'w' must have already been initialized with \verb'GrB_Vector_new', and it
must have no entries in it before calling \verb'GrB_Vector_build'.
This function is just like \verb'GrB_Matrix_build' (see
Section~\ref{matrix_build}), except that it builds a sparse vector instead of a
sparse matrix.  For a description of what \verb'GrB_Vector_build' does, refer
to \verb'GrB_Matrix_build'.  For a vector, the list of column indices \verb'J'
in \verb'GrB_Matrix_build' is implicitly a vector of length \verb'nvals' all
equal to zero.  Otherwise the methods are identical.

If \verb'dup' is \verb'NULL', any duplicates result in an error.
If \verb'dup' is the special binary operator \verb'GxB_IGNORE_DUP', then
any duplicates are ignored.  If duplicates appear, the last one in the
list of tuples is taken and the prior ones ignored.  This is not an error.
%
The \verb'dup' operator cannot be a binary operator
created by \verb'GxB_BinaryOp_new_IndexOp'.

\begin{alert}
{\bf SPEC:} Results are defined even if \verb'dup' is non-associative and/or
non-commutative.
\end{alert}

%-------------------------------------------------------------------------------
\subsubsection{{\sf GrB\_Vector\_build\_Vector:} build a vector from a set of tuples}
%-------------------------------------------------------------------------------
\label{vector_build_Vector}

\begin{mdframed}[userdefinedwidth=6in]
{\footnotesize
\begin{verbatim}
GrB_Info GrB_Vector_build       // build a vector from (I,X) tuples
(
    GrB_Vector w,               // vector to build
    const GrB_Vector I_vector,  // row indices
    const GrB_Vector X_vector,  // values
    const GrB_BinaryOp dup,     // binary function to assemble duplicates
    const GrB_Descriptor desc
) ;
\end{verbatim}
} \end{mdframed}

\verb'GxB_Vector_build_Vector' is identical to \verb'GrB_Vector_build', except
that the inputs \verb'I' and \verb'X' are \verb'GrB_Vector' objects, each with
\verb'nvals' entries.  The interpretation of \verb'I_vector' and
\verb'X_vector' are controlled by descriptor settings \verb'GxB_ROWINDEX_LIST'
and \verb'GxB_VALUE_LIST', respectively.  The method can use either the indices
or values of each of the input vectors; the default is to use the values.  See
Section~\ref{ijxvector} for details.

%-------------------------------------------------------------------------------
\subsubsection{{\sf GxB\_Vector\_build\_Scalar:} build a vector from a set of tuples}
%-------------------------------------------------------------------------------
\label{vector_build_Scalar}

\begin{mdframed}[userdefinedwidth=6in]
{\footnotesize
\begin{verbatim}
GrB_Info GrB_Vector_build       // build a vector from (I,scalar) tuples
(
    GrB_Vector w,                   // vector to build
    const GrB_Index *I,             // array of row indices of tuples
    GrB_Scalar scalar,              // value for all tuples
    GrB_Index nvals                 // number of tuples
) ;
\end{verbatim} } \end{mdframed}

\verb'GxB_Vector_build_Scalar' constructs a sparse vector \verb'w' from a set
of tuples defined by the index array \verb'I' of length \verb'nvals', and a
scalar.  The scalar is the value of all of the tuples.  Unlike
\verb'GrB_Vector_build', there is no \verb'dup' operator to handle duplicate
entries.  Instead, any duplicates are silently ignored (if the number of
duplicates is desired, simply compare the input \verb'nvals' with the value
returned by \verb'GrB_Vector_nvals' after the vector is constructed).  All
entries in the sparsity pattern of \verb'w' are identical, and equal to the
input scalar value.

%-------------------------------------------------------------------------------
\subsubsection{{\sf GxB\_Vector\_build\_Scalar\_Vector:} build a vector from a set of tuples}
%-------------------------------------------------------------------------------
\label{vector_build_Scalar_Vector}

\begin{mdframed}[userdefinedwidth=6in]
{\footnotesize
\begin{verbatim}
GrB_Info GrB_Vector_build       // build a vector from (I,scalar) tuples
(
    GrB_Vector w,               // vector to build
    const GrB_Vector I_vector,  // row indices
    const GrB_Scalar scalar,    // value for all tuples
    const GrB_Descriptor desc
) ;
\end{verbatim} } \end{mdframed}

\verb'GxB_Vector_build_Scalar_Vector' is identical to
\verb'GxB_Vector_build_Scalar', except that the inputs \verb'I' and \verb'X'
are \verb'GrB_Vector' objects, each with \verb'nvals' entries.  The
interpretation of \verb'I_vector' is controlled by the descriptor setting
\verb'GxB_ROWINDEX_LIST'.  The method
can use either the indices or values of the \verb'I_input' vector; the default
is to use the values.  See Section~\ref{ijxvector} for details.

%-------------------------------------------------------------------------------
\subsubsection{{\sf GrB\_Vector\_setElement:}    add an entry to a vector}
%-------------------------------------------------------------------------------
\label{vector_setElement}

\begin{mdframed}[userdefinedwidth=6in]
{\footnotesize
\begin{verbatim}
GrB_Info GrB_Vector_setElement          // w(i) = x
(
    GrB_Vector w,                       // vector to modify
    <type> x,                           // scalar to assign to w(i)
    GrB_Index i                         // index
) ;
\end{verbatim} } \end{mdframed}

\verb'GrB_Vector_setElement' sets a single entry in a vector, \verb'w(i) = x'.
The operation is exactly like setting a single entry in an \verb'n'-by-1
matrix, \verb'A(i,0) = x', where the column index for a vector is implicitly
\verb'j=0'.  For further details of this function, see
\verb'GrB_Matrix_setElement' in Section~\ref{matrix_setElement}.
If an error occurs, \verb'GrB_error(&err,w)' returns details about the error.

\newpage
%-------------------------------------------------------------------------------
\subsubsection{{\sf GrB\_Vector\_extractElement:} get an entry from a vector}
%-------------------------------------------------------------------------------
\label{vector_extractElement}

\begin{mdframed}[userdefinedwidth=6in]
{\footnotesize
\begin{verbatim}
GrB_Info GrB_Vector_extractElement  // x = v(i)
(
    <type> *x,                  // scalar extracted (non-opaque, C scalar)
    const GrB_Vector v,         // vector to extract an entry from
    GrB_Index i                 // index
) ;

GrB_Info GrB_Vector_extractElement  // x = v(i)
(
    GrB_Scalar x,               // GrB_Scalar extracted
    const GrB_Vector v,         // vector to extract an entry from
    GrB_Index i                 // index
) ;
\end{verbatim} } \end{mdframed}

\verb'GrB_Vector_extractElement' extracts a single entry from a vector,
\verb'x = v(i)'.  The method is identical to extracting a single entry
\verb'x = A(i,0)' from an \verb'n'-by-1 matrix; see
Section~\ref{matrix_extractElement}.

%-------------------------------------------------------------------------------
\subsubsection{{\sf GxB\_Vector\_isStoredElement:} check if entry present in vector}
%-------------------------------------------------------------------------------
\label{vector_isStoredElement}

\begin{mdframed}[userdefinedwidth=6in]
{\footnotesize
\begin{verbatim}
GrB_Info GxB_Vector_isStoredElement
(
    const GrB_Vector v,         // check presence of entry v(i)
    GrB_Index i                 // index
) ;
\end{verbatim} } \end{mdframed}

\verb'GxB_Vector_isStoredElement' checks if a single entry \verb'v(i)'
is present, returning \verb'GrB_SUCCESS' if the entry is present or
\verb'GrB_NO_VALUE' otherwise.  The value of \verb'v(i)' is not returned.
See also Section~\ref{matrix_isStoredElement}.

%-------------------------------------------------------------------------------
\subsubsection{{\sf GrB\_Vector\_removeElement:} remove an entry from a vector}
%-------------------------------------------------------------------------------
\label{vector_removeElement}

\begin{mdframed}[userdefinedwidth=6in]
{\footnotesize
\begin{verbatim}
GrB_Info GrB_Vector_removeElement
(
    GrB_Vector w,                   // vector to remove an entry from
    GrB_Index i                     // index
) ;
\end{verbatim} } \end{mdframed}

\verb'GrB_Vector_removeElement' removes a single entry \verb'w(i)' from a vector.
If no entry is present at \verb'w(i)', then the vector is not modified.
If an error occurs, \verb'GrB_error(&err,w)' returns details about the error.

%-------------------------------------------------------------------------------
\subsubsection{{\sf GrB\_Vector\_extractTuples:} get all entries from a vector}
%-------------------------------------------------------------------------------
\label{vector_extractTuples}

\begin{mdframed}[userdefinedwidth=6in]
{\footnotesize
\begin{verbatim}
GrB_Info GrB_Vector_extractTuples           // [I,~,X] = find (v)
(
    GrB_Index *I,               // array for returning row indices of tuples
    <type> *X,                  // array for returning values of tuples
    GrB_Index *nvals,           // I, X size on input; # tuples on output
    const GrB_Vector v          // vector to extract tuples from
) ;
\end{verbatim} } \end{mdframed}

\verb'GrB_Vector_extractTuples' extracts all tuples from a sparse vector,
analogous to \verb'[I,~,X] = find(v)' in MATLAB/Octave.  This function is
identical to its \verb'GrB_Matrix_extractTuples' counterpart, except that the
array of column indices \verb'J' does not appear in this function.  Refer to
Section~\ref{matrix_extractTuples} where further details of this function are
described.

%-------------------------------------------------------------------------------
\subsubsection{{\sf GxB\_Vector\_extractTuples\_Vector:} get all entries from a vector}
%-------------------------------------------------------------------------------
\label{vector_extractTuples_Vector}

\begin{mdframed}[userdefinedwidth=6in]
{\footnotesize
\begin{verbatim}
GrB_Info GrB_Vector_extractTuples           // [I,~,X] = find (v)
(
    GrB_Vector I_vector,    // row indices
    GrB_Vector X_vector,    // values
    const GrB_Vector V,     // vectors to extract tuples from
    const GrB_Descriptor desc   // currently unused; for future expansion
) ;

\end{verbatim} } \end{mdframed}

\verb'GxB_Vector_extractTuples_Vector' is identical to
\verb'GrB_Vector_extractTuples' except that its two outputs are
\verb'GrB_Vector' objects.  The vectors \verb'I_vector' and \verb'X_vector'
objects must exist on input.  On output, any prior content is erased and
their type, dimensions, and values are revised to contain dense vectors of
length \verb'nvals'.

\newpage
%-------------------------------------------------------------------------------
\subsubsection{{\sf GrB\_Vector\_resize:}          resize a vector}
%-------------------------------------------------------------------------------
\label{vector_resize}

\begin{mdframed}[userdefinedwidth=6in]
{\footnotesize
\begin{verbatim}
GrB_Info GrB_Vector_resize      // change the size of a vector
(
    GrB_Vector u,               // vector to modify
    GrB_Index nrows_new         // new number of rows in vector
) ;
\end{verbatim} } \end{mdframed}

\verb'GrB_Vector_resize' changes the size of a vector.  If the dimension
decreases, entries that fall outside the resized vector are deleted.

%-------------------------------------------------------------------------------
\subsubsection{{\sf GxB\_Vector\_diag:} extract a diagonal from a matrix}
%-------------------------------------------------------------------------------
\label{vector_diag}

\begin{mdframed}[userdefinedwidth=6in]
{\footnotesize
\begin{verbatim}
GrB_Info GxB_Vector_diag    // extract a diagonal from a matrix
(
    GrB_Vector v,                   // output vector
    const GrB_Matrix A,             // input matrix
    int64_t k,
    const GrB_Descriptor desc       // unused, except threading control
) ;
\end{verbatim} } \end{mdframed}


\verb'GxB_Vector_diag' extracts a vector \verb'v' from an input matrix
\verb'A', which may be rectangular.  If \verb'k' = 0, the main diagonal of
\verb'A' is extracted; \verb'k' $> 0$ denotes diagonals above the main diagonal
of \verb'A', and \verb'k' $< 0$ denotes diagonals below the main diagonal of
\verb'A'.  Let \verb'A' have dimension $m$-by-$n$.  If \verb'k' is in the range
0 to $n-1$, then \verb'v' has length $\min(m,n-k)$.  If \verb'k' is negative
and in the range -1 to $-m+1$, then \verb'v' has length $\min(m+k,n)$.  If
\verb'k' is outside these ranges, \verb'v' has length 0 (this is not an error).
This function computes the same thing as the MATLAB/Octave statement
\verb'v=diag(A,k)' when \verb'A' is a matrix, except that
\verb'GxB_Vector_diag' can also do typecasting.

The vector \verb'v' must already exist on input, and
\verb'GrB_Vector_size (&len,v)' must return \verb'len' = 0 if \verb'k' $\ge n$
or \verb'k' $\le -m$, \verb'len' $=\min(m,n-k)$ if \verb'k' is in the range 0
to $n-1$, and \verb'len' $=\min(m+k,n)$ if \verb'k' is in the range -1 to
$-m+1$.  Any existing entries in \verb'v' are discarded.  The type of \verb'v'
is preserved, so that if the type of \verb'A' and \verb'v' differ, the entries
are typecasted into the type of \verb'v'.  Any settings made to \verb'v' by
\verb'GrB_set' (bitmap switch and sparsity control) are
unchanged.

\newpage

%-------------------------------------------------------------------------------
\subsubsection{{\sf GxB\_Vector\_memoryUsage:} memory used by a vector}
%-------------------------------------------------------------------------------
\label{vector_memusage}

\begin{mdframed}[userdefinedwidth=6in]
{\footnotesize
\begin{verbatim}
GrB_Info GxB_Vector_memoryUsage  // return # of bytes used for a vector
(
    size_t *size,           // # of bytes used by the vector v
    const GrB_Vector v      // vector to query
) ;
\end{verbatim} } \end{mdframed}

Returns the memory space required for a vector, in bytes.
By default, any read-only components are not included in the total memory.
This can be changed with via \verb'GrB_get'; see Section~\ref{get_set_global}.

%-------------------------------------------------------------------------------
\subsubsection{{\sf GxB\_Vector\_type:} type of a vector}
%-------------------------------------------------------------------------------
\label{vector_type}

\begin{mdframed}[userdefinedwidth=6in]
{\footnotesize
\begin{verbatim}
GrB_Info GxB_Vector_type    // get the type of a vector
(
    GrB_Type *type,         // returns the type of the vector
    const GrB_Vector v      // vector to query
) ;
\end{verbatim} } \end{mdframed}

Returns the type of a vector.  See \verb'GxB_Matrix_type' for details
(Section~\ref{matrix_type}).

%-------------------------------------------------------------------------------
\subsubsection{{\sf GrB\_Vector\_free:}          free a vector}
%-------------------------------------------------------------------------------
\label{vector_free}

\begin{mdframed}[userdefinedwidth=6in]
{\footnotesize
\begin{verbatim}
GrB_Info GrB_free           // free a vector
(
    GrB_Vector *v           // handle of vector to free
) ;
\end{verbatim}
} \end{mdframed}

\verb'GrB_Vector_free' frees a vector.  Either usage:

    {\small
    \begin{verbatim}
    GrB_Vector_free (&v) ;
    GrB_free (&v) ; \end{verbatim}}

\noindent
frees the vector \verb'v' and sets \verb'v' to \verb'NULL'.  It safely does
nothing if passed a \verb'NULL' handle, or if \verb'v == NULL' on input.  Any
pending updates to the vector are abandoned.


