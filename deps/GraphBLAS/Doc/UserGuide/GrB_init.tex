
\newpage
%%%%%%%%%%%%%%%%%%%%%%%%%%%%%%%%%%%%%%%%%%%%%%%%%%%%%%%%%%%%%%%%%%%%%%%%%%%%%%%%
\section{GraphBLAS Initialization/Finalization} %%%%%%%%%%%%%%%%%%%%%%%%%%%%%%%%
%%%%%%%%%%%%%%%%%%%%%%%%%%%%%%%%%%%%%%%%%%%%%%%%%%%%%%%%%%%%%%%%%%%%%%%%%%%%%%%%
\label{init_and_fini}

A user application that directly relies on GraphBLAS must include the
\verb'GraphBLAS.h' header file:

\begin{mdframed}[userdefinedwidth=6in]
{\footnotesize
\begin{verbatim}
    #include "GraphBLAS.h"
\end{verbatim}
} \end{mdframed}

The \verb'GraphBLAS.h' file defines functions, types, and macros prefixed with
\verb'GrB_' and \verb'GxB_' that may be used in user applications.  The prefix
\verb'GrB_' denotes items that appear in the official {\em GraphBLAS C API
Specification}.  The prefix \verb'GxB_' refers to SuiteSparse-specific
extensions to the GraphBLAS API.

The \verb'GraphBLAS.h' file includes all the definitions required to use
GraphBLAS, including the following macros that can assist a user application in
compiling and using GraphBLAS.

There are two version numbers associated with SuiteSparse:GraphBLAS:
the version of the {\em GraphBLAS C API Specification} it
conforms to, and the version of the implementation itself.  These can
be used in the following manner in a user application:

{\footnotesize
\begin{verbatim}
    #if GxB_SPEC_VERSION >= GxB_VERSION (2,0,3)
    ... use features in GraphBLAS specification 2.0.3 ...
    #else
    ... only use features in early specifications
    #endif

    #if GxB_IMPLEMENTATION >= GxB_VERSION (5,2,0)
    ... use features from version 5.2.0 (or later)
    of a specific GraphBLAS implementation
    #endif \end{verbatim}}

SuiteSparse:GraphBLAS also defines the following strings with \verb'#define'.
Refer to the \verb'GraphBLAS.h' file for details.

\vspace{0.2in}
{\footnotesize
\begin{tabular}{ll}
\hline
Macro                & purpose                                      \\
\hline
\verb'GxB_IMPLEMENTATION_ABOUT'
    & this particular implementation, copyright, and URL \\
\verb'GxB_IMPLEMENTATION_DATE'
    & the date of this implementation \\
\verb'GxB_SPEC_ABOUT'
    & the GraphBLAS specification for this implementation \\
\verb'GxB_SPEC_DATE'
    & the date of the GraphBLAS specification \\
\verb'GxB_IMPLEMENTATION_LICENSE'
    & the license for this particular implementation \\
\hline
\end{tabular}
}
\vspace{0.2in}

Finally, SuiteSparse:GraphBLAS gives itself a unique name of the form
\verb'GxB_SUITESPARSE_GRAPHBLAS' that the user application can use in
\verb'#ifdef' tests. This is helpful in case a particular implementation
provides non-standard features that extend the GraphBLAS specification, such as
additional predefined built-in operators, or if a GraphBLAS implementation does
not yet fully implement all of the GraphBLAS specification.

For example, SuiteSparse:GraphBLAS predefines additional built-in operators not
in the specification.  If the user application wishes to use these in any
GraphBLAS implementation, an \verb'#ifdef' can control when they are used.
Refer to the examples in the \verb'GraphBLAS/Demo' folder.

As another example, the GraphBLAS API states that an
implementation need not define the order in which \verb'GrB_Matrix_build'
assembles duplicate tuples in its \verb'[I,J,X]' input arrays.  As a result, no
particular ordering should be relied upon in general.  However,
SuiteSparse:GraphBLAS does guarantee an ordering, and this guarantee will be
kept in future versions of SuiteSparse:GraphBLAS as well.  Since not all
implementations will ensure a particular ordering, the following can be used to
exploit the ordering returned by SuiteSparse:GraphBLAS.

    {\footnotesize
    \begin{verbatim}
    #ifdef GxB_SUITESPARSE_GRAPHBLAS
    // duplicates in I, J, X assembled in a specific order;
    // results are well-defined even if op is not associative.
    GrB_Matrix_build (C, I, J, X, nvals, op) ;
    #else
    // duplicates in I, J, X assembled in no particular order;
    // results are undefined if op is not associative.
    GrB_Matrix_build (C, I, J, X, nvals, op) ;
    #endif \end{verbatim}}

The remainder of this section describes GraphBLAS functions that start or finalize GraphBLAS,
error handling, and the GraphBLAS integer.

\vspace{0.2in}
{\footnotesize
\begin{tabular}{lll}
\hline
GraphBLAS function/type   & purpose                                 & Section \\
\hline
\verb'GrB_Index'     & the GraphBLAS integer                        & \ref{grbindex} \\
\verb'GrB_init'      & start up GraphBLAS                           & \ref{init} \\
\verb'GrB_getVersion'& C API supported by the library               & \ref{getVersion} \\
\verb'GxB_init'      & start up GraphBLAS with different \verb'malloc' & \ref{xinit} \\
\verb'GrB_Info'      & status code returned by GraphBLAS functions  & \ref{info} \\
\verb'GrB_error'     & get more details on the last error           & \ref{error} \\
\verb'GrB_finalize'  & finish GraphBLAS                             & \ref{finalize} \\
\hline
\end{tabular}
}
\vspace{0.2in}

%===============================================================================
\subsection{{\sf GrB\_Index:} the GraphBLAS integer} %==========================
%===============================================================================
\label{grbindex}

Matrix and vector dimensions and indexing rely on a specific integer,
\verb'GrB_Index', which is defined in \verb'GraphBLAS.h' as

    {\footnotesize
    \begin{verbatim}
    typedef uint64_t GrB_Index ; \end{verbatim}}

Row and column indices of an \verb'nrows'-by-\verb'ncols' matrix range from
zero to the \verb'nrows-1' for the rows, and zero to \verb'ncols-1' for the
columns.  Indices are zero-based, like C, and not one-based, like
MATLAB/Octave.  In SuiteSparse:GraphBLAS, the largest permitted index value
is \verb'GrB_INDEX_MAX', defined as $2^{60}-1$.  The largest permitted
matrix or vector dimension is $2^{60}$ (that is, \verb'GrB_INDEX_MAX+1').
The largest \verb'GrB_Matrix' that
SuiteSparse: GraphBLAS can construct is thus $2^{60}$-by-$2^{60}$.  An
$n$-by-$n$ matrix $\bf A$ that size can easily be constructed in practice with
$O(|{\bf A}|)$ memory requirements, where $|{\bf A}|$ denotes the number of
entries that explicitly appear in the pattern of ${\bf A}$.  The time and
memory required to construct a matrix that large does not depend on $n$, since
SuiteSparse:GraphBLAS can represent ${\bf A}$ in hypersparse form (see
Section~\ref{hypersparse}).  The largest \verb'GrB_Vector' that can be
constructed is $2^{60}$-by-1.

Internally, GraphBLAS may store its integer indices using 32-bit integers
(as of GraphBLAS v10.0.0).  See Section~\ref{integer_bits} for details.

%===============================================================================
\subsection{{\sf GrB\_init:} initialize GraphBLAS} %============================
%===============================================================================
\label{init}

\begin{mdframed}[userdefinedwidth=6in]
{\footnotesize
\begin{verbatim}
typedef enum
{
    GrB_NONBLOCKING = 0,    // methods may return with pending computations
    GrB_BLOCKING = 1        // no computations are ever left pending
}
GrB_Mode ;
\end{verbatim}
}\end{mdframed}

\begin{mdframed}[userdefinedwidth=6in]
{\footnotesize
\begin{verbatim}
GrB_Info GrB_init           // start up GraphBLAS
(
    int mode                // blocking or non-blocking mode (GrB_Mode)
) ;
\end{verbatim}
}\end{mdframed}

\hypertarget{link:init}{\mbox{ }}%
\verb'GrB_init' must be called before any other GraphBLAS operation.  It
defines the mode that GraphBLAS will use:  blocking or non-blocking.  With
blocking mode, all operations finish before returning to the user application.
With non-blocking mode, operations can be left pending, and are computed only
when needed.  Non-blocking mode can be much faster than blocking mode, by many
orders of magnitude in extreme cases.  Blocking mode should be used only when
debugging a user application.  The mode cannot be changed once it is set by
\verb'GrB_init'.

GraphBLAS objects are opaque.  This allows GraphBLAS to
postpone operations and then do them later in a more efficient manner by
rearranging them and grouping them together.  In non-blocking mode, the
computations required to construct an opaque GraphBLAS object might not be
finished when the GraphBLAS method or operation returns to the user.  However,
user-provided arrays are not opaque, and GraphBLAS methods and operations that
read them (such as \verb'GrB_Matrix_build') or write to them (such as
\verb'GrB_Matrix_extractTuples') always finish reading them, or creating them,
when the method or operation returns to the user application.

All methods and operations that extract values from a GraphBLAS object and
return them into non-opaque user arrays always ensure that the user-visible
arrays are fully populated when they return: \verb'GrB_*_reduce' (to scalar),
\verb'GrB_*_nvals', \verb'GrB_*_extractElement', and
\verb'GrB_*_extractTuples'.  These functions do {\em not} guarantee that the
opaque objects they depend on are finalized.  To do that, use
\verb'GrB_wait' instead.

SuiteSparse:GraphBLAS is multithreaded internally, via OpenMP, and it is also
safe to use in a multithreaded user application.  See Section~\ref{sec:install}
for details.
User threads must not operate on the same matrices at the same time, with one
exception.  Multiple user threads can use the same matrices or vectors as
inputs to GraphBLAS operations or methods, but only if they have no
pending operations (use \verb'GrB_wait'
first).  User threads cannot simultaneously modify a matrix or vector via any
GraphBLAS operation or method.

It is safe to use the internal parallelism in SuiteSparse:GraphBLAS on
matrices, vectors, and scalars that are not yet completed.  The library
handles this on its own.  The \verb'GrB_wait' function is only
needed when a user application makes multiple calls to GraphBLAS in parallel,
from multiple user threads.

With multiple user threads, exactly one user thread must call \verb'GrB_init'
before any user thread may call any \verb'GrB_*' or \verb'GxB_*' function.
When the user application is finished, exactly one user thread must call
\verb'GrB_finalize', after which no user thread may call any \verb'GrB_*' or
\verb'GxB_*' function.
The mode of a GraphBLAS session can be queried with \verb'GrB_get';
see Section~\ref{options} for details.

%===============================================================================
\subsection{{\sf GrB\_getVersion:} determine the C API Version} %===============
%===============================================================================
\label{getVersion}

\begin{mdframed}[userdefinedwidth=6in]
{\footnotesize
\begin{verbatim}
GrB_Info GrB_getVersion         // run-time access to C API version number
(
    unsigned int *version,      // returns GRB_VERSION
    unsigned int *subversion    // returns GRB_SUBVERSION
) ;
\end{verbatim}
}\end{mdframed}

GraphBLAS defines two compile-time constants that
define the version of the C API Specification
that is implemented by the library:
\verb'GRB_VERSION' and \verb'GRB_SUBVERSION'.
If the user program was compiled with one
version of the library but linked with a different one later on, the
compile-time version check with \verb'GRB_VERSION' would be stale.
\verb'GrB_getVersion' thus provides a run-time access of the version of the C
API Specification supported by the library.

%===============================================================================
\subsection{{\sf GxB\_init:} initialize with alternate malloc} %================
%===============================================================================
\label{xinit}

\begin{mdframed}[userdefinedwidth=6in]
{\footnotesize
\begin{verbatim}
GrB_Info GxB_init           // start up GraphBLAS and also define malloc
(
    int mode,               // blocking or non-blocking mode (GrB_Mode)
    // pointers to memory management functions.
    void * (* user_malloc_func  ) (size_t),
    void * (* user_calloc_func  ) (size_t, size_t),
    void * (* user_realloc_func ) (void *, size_t),
    void   (* user_free_func    ) (void *)
) ;
\end{verbatim}
}\end{mdframed}

\verb'GxB_init' is identical to \verb'GrB_init', except that it also redefines
the memory management functions that SuiteSparse:GraphBLAS will use.  Giving
the user application control over this is particularly important when using the
\verb'GxB_*serialize' and \verb'GxB_Container' methods described in Section
\ref{serialize_deserialize} and \ref{container}, since they require the user
application and GraphBLAS to use the same memory manager.
\verb'user_calloc_func' and \verb'user_realloc_func' are optional, and
may be \verb'NULL'.  If \verb'NULL', then the \verb'user_malloc_func' is
relied on instead, for all memory allocations.
These functions can only be set once, when GraphBLAS starts.
They can be queried using \verb'GrB_get' (see
Section~\ref{get_set_global}).
Either
\verb'GrB_init' or \verb'GxB_init' must be called before any other GraphBLAS
operation, but not both.  The functions passed to \verb'GxB_init' must be
thread-safe.
The following usage is identical to \verb'GrB_init(mode)':

    {\footnotesize
    \begin{verbatim}
    GxB_init (mode, malloc, calloc, realloc, free) ; \end{verbatim}}

%===============================================================================
\subsection{{\sf GrB\_Info:} status code returned by GraphBLAS} %===============
%===============================================================================
\label{info}

Each GraphBLAS method and operation returns its status to the caller as its
return value, an enumerated type (an \verb'enum') called \verb'GrB_Info'.  The
first two values in the following table denote a successful status, the rest
are error codes.

Not all GraphBLAS methods or operations can return all status codes.
In the discussions of each method and operation in this User Guide, most of the
obvious error code returns are not discussed.  For example, if a required input
is a \verb'NULL' pointer, then \verb'GrB_NULL_POINTER' is returned.  Only error
codes specific to the method or that require elaboration are discussed here.
For a full list of the status codes that each GraphBLAS function can return,
refer to {\em The GraphBLAS C API Specification} \cite{spec,spec2}.

\vspace{0.2in}
\noindent
{\small
\begin{tabular}{lrp{2.8in}}
\hline
Error                         & value & description \\
\hline
\verb'GrB_SUCCESS'              & 0   & the method or operation was successful \\
\verb'GrB_NO_VALUE'             & 1   & the method was successful, but the entry \\
                                &     & does not appear in the matrix or vector. \\
\verb'GxB_EXHAUSTED'            & 2   & the iterator is exhausted \\
\hline
\hline
\verb'GrB_UNINITIALIZED_OBJECT' & -1   & object has not been initialized \\
\verb'GrB_NULL_POINTER'         & -2   & input pointer is \verb'NULL' \\
\verb'GrB_INVALID_VALUE'        & -3   & generic error code; some value is bad \\
\verb'GrB_INVALID_INDEX'        & -4   & a row or column index is out of bounds \\
\verb'GrB_DOMAIN_MISMATCH'      & -5   & object domains are not compatible \\
\verb'GrB_DIMENSION_MISMATCH'   & -6   & matrix dimensions do not match \\
\verb'GrB_OUTPUT_NOT_EMPTY'     & -7   & output matrix already has values in it \\
\verb'GrB_NOT_IMPLEMENTED'      & -8   & not implemented in SS:GrB \\
\verb'GrB_ALREADY_SET'          & -9   & field already written to \\
\verb'GrB_PANIC'                & -101 & unrecoverable error \\
\verb'GrB_OUT_OF_MEMORY'        & -102 & out of memory \\
\verb'GrB_INSUFFICIENT_SPACE'   & -103 & output array not large enough \\
\verb'GrB_INVALID_OBJECT'       & -104 & object is corrupted \\
\verb'GrB_INDEX_OUT_OF_BOUNDS'  & -105 & a row or column index is out of bounds \\
\verb'GrB_EMPTY_OBJECT'         & -106 & a input scalar has no entry \\
\verb'GxB_JIT_ERROR'            &-7001 & JIT compiler error \\
\verb'GxB_OUTPUT_IS_READONLY'   &-7003 & output has read-only components \\
\hline
\end{tabular}
\vspace{0.2in}
}

%===============================================================================
\subsection{{\sf GrB\_error:} get more details on the last error} %=============
%===============================================================================
\label{error}

\begin{mdframed}[userdefinedwidth=6in]
{\footnotesize
\begin{verbatim}
GrB_Info GrB_error      // return a string describing the last error
(
    const char **error, // error string
    <type> object       // a GrB_matrix, GrB_Vector, etc.
) ;
\end{verbatim}
}\end{mdframed}

Each GraphBLAS method and operation returns a \verb'GrB_Info' error code.  The
\verb'GrB_error' function returns additional information on the error for a
particular object in a null-terminated string.  The string returned by
\verb'GrB_error' is never a \verb'NULL' string, but it may have length zero
(with the first entry being the \verb"'\0'" string-termination value).  The
string must not be freed or modified.

    {\footnotesize
    \begin{verbatim}
    info = GrB_some_method_here (C, ...) ;
    if (! (info == GrB_SUCCESS || info == GrB_NO_VALUE))
    {
        char *err ;
        GrB_error (&err, C) ;
        printf ("info: %d error: %s\n", info, err) ;
    } \end{verbatim}}

If \verb'C' has no error status, or if the error is not recorded in
the string, an empty non-null string is returned.  In particular,
out-of-memory conditions result in an empty string from \verb'GrB_error'.

SuiteSparse:GraphBLAS reports many helpful details via \verb'GrB_error'.  For
example, if a row or column index is out of bounds, the report will state what
those bounds are.  If a matrix dimension is incorrect, the mismatching
dimensions will be provided.  Refer to
the output of the example programs in the \verb'Demo' and \verb'Test' folder,
which intentionally generate errors to illustrate the use of \verb'GrB_error'.

The only functions in GraphBLAS that return an error string are functions that
have a single input/output argument \verb'C', as a \verb'GrB_Matrix',
\verb'GrB_Vector', \verb'GrB_Scalar', or \verb'GrB_Descriptor'. Methods that
create these objects (such as \verb'GrB_Matrix_new') return a \verb'NULL'
object on failure, so these methods cannot also return an error string in
\verb'C'.

Any subsequent GraphBLAS method that modifies the object \verb'C' clears the
error string.

Note that \verb'GrB_NO_VALUE' is an not error, but an informational status.
\verb'GrB_*_extractElment(&x,A,i,j)', which does \verb'x=A(i,j)', returns this
value to indicate that \verb'A(i,j)' is not present in the matrix.  That
method does not have an input/output object so it cannot return an error
string.

%===============================================================================
\subsection{{\sf GrB\_finalize:} finish GraphBLAS} %============================
%===============================================================================
\label{finalize}

\begin{mdframed}[userdefinedwidth=6in]
{\footnotesize
\begin{verbatim}
GrB_Info GrB_finalize ( ) ;     // finish GraphBLAS
\end{verbatim}
}\end{mdframed}

\verb'GrB_finalize' must be called as the last GraphBLAS operation, even after
all calls to \verb'GrB_free'.  All GraphBLAS objects created by the user
application should be freed first, before calling \verb'GrB_finalize' since
\verb'GrB_finalize' will not free those objects.  In non-blocking mode,
GraphBLAS may leave some computations as pending.  These computations can be
safely abandoned if the user application frees all GraphBLAS objects it has
created and then calls \verb'GrB_finalize'.  When the user application is
finished, exactly one user thread must call \verb'GrB_finalize'.

