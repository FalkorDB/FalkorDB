
\newpage
%===============================================================================
\subsection{The GraphBLAS type: {\sf GrB\_Type}} %==============================
%===============================================================================
\label{type}

A GraphBLAS \verb'GrB_Type' defines the type of scalar values that a matrix or
vector contains, and the type of scalar operands for a unary or binary
operator.  There are 13 built-in types, and a user application can define
any types of its own as well.  The built-in types correspond to built-in types
in C (in the \verb'#include' files \verb'stdbool.h', \verb'stdint.h', and
\verb'complex.h') as listed in the following table.

\vspace{0.2in}
\noindent
{\footnotesize
\begin{tabular}{llll}
\hline
GraphBLAS         & C type           & description              & range \\
type              &                  &                          & \\
\hline
\verb'GrB_BOOL'   & \verb'bool'      & Boolean                  & true (1), false (0) \\
\hline
\verb'GrB_INT8'   & \verb'int8_t'    & 8-bit signed integer     & -128 to 127 \\
\verb'GrB_INT16'  & \verb'int16_t'   & 16-bit integer           & $-2^{15}$ to $2^{15}-1$ \\
\verb'GrB_INT32'  & \verb'int32_t'   & 32-bit integer           & $-2^{31}$ to $2^{31}-1$ \\
\verb'GrB_INT64'  & \verb'int64_t'   & 64-bit integer           & $-2^{63}$ to $2^{63}-1$ \\
\hline
\verb'GrB_UINT8'  & \verb'uint8_t'   & 8-bit unsigned integer   & 0 to 255 \\
\verb'GrB_UINT16' & \verb'uint16_t'  & 16-bit unsigned integer  & 0 to $2^{16}-1$ \\
\verb'GrB_UINT32' & \verb'uint32_t'  & 32-bit unsigned integer  & 0 to $2^{32}-1$ \\
\verb'GrB_UINT64' & \verb'uint64_t'  & 64-bit unsigned integer  & 0 to $2^{64}-1$ \\
\hline
\verb'GrB_FP32'   & \verb'float'     & 32-bit IEEE 754          & \verb'-Inf' to \verb'+Inf'\\
\verb'GrB_FP64'   & \verb'double'    & 64-bit IEEE 754          & \verb'-Inf' to \verb'+Inf'\\
\hline
\verb'GxB_FC32'   & \verb'float complex'  & 32-bit complex & \verb'-Inf' to \verb'+Inf'\\
\verb'GxB_FC64'   & \verb'double complex' & 64-bit complex & \verb'-Inf' to \verb'+Inf'\\
\hline
\end{tabular}
}
\vspace{0.2in}

The C11 definitions of \verb'float complex' and \verb'double complex'
are not always available.  The \verb'GraphBLAS.h' header defines them as
\verb'GxB_FC32_t' and \verb'GxB_FC64_t', respectively.

The user application can also define new types based on any \verb'typedef' in
the C language whose values are held in a contiguous region of memory of fixed
size.  For example, a user-defined \verb'GrB_Type' could be created to hold any
C \verb'struct' whose content is self-contained.  A C \verb'struct' containing
pointers might be problematic because GraphBLAS would not know to dereference
the pointers to traverse the entire ``scalar'' entry, but this can be done if
the objects referenced by these pointers are not moved.  A user-defined complex
type with real and imaginary types can be defined, or even a ``scalar'' type
containing a fixed-sized dense matrix (see Section~\ref{type_new}).  The
possibilities are endless.  GraphBLAS can create and operate on sparse matrices
and vectors in any of these types, including any user-defined ones.  For
user-defined types, GraphBLAS simply moves the data around itself (via
\verb'memcpy'), and then passes the values back to user-defined functions when
it needs to do any computations on the type.  The next sections describe the
methods for the \verb'GrB_Type' object:

\vspace{0.2in}
{\footnotesize
\begin{tabular}{lll}
\hline
GraphBLAS function       & purpose                          & Section \\
\hline
\verb'GrB_Type_new'      & create a user-defined type       & \ref{type_new} \\
\verb'GxB_Type_new'      & create a user-defined type,
                            with name and definition        & \ref{type_new_named} \\
\verb'GrB_Type_wait'     & wait for a user-defined type     & \ref{type_wait} \\
\verb'GxB_Type_from_name'& return the type from its name    & \ref{type_from_name} \\
\verb'GrB_Type_free'     & free a user-defined type         & \ref{type_free} \\
\verb'GrB_get'           & get properties of a type         & \ref{get_set_type} \\
\verb'GrB_set'           & set the type name/definitition   & \ref{get_set_type} \\
\hline
\end{tabular}
}

%-------------------------------------------------------------------------------
\subsubsection{{\sf GrB\_Type\_new:} create a user-defined type}
%-------------------------------------------------------------------------------
\label{type_new}

\begin{mdframed}[userdefinedwidth=6in]
{\footnotesize
\begin{verbatim}
GrB_Info GrB_Type_new           // create a new GraphBLAS type
(
    GrB_Type *type,             // handle of user type to create
    size_t sizeof_ctype         // size = sizeof (ctype) of the C type
) ;
\end{verbatim}
}\end{mdframed}

\verb'GrB_Type_new' creates a new user-defined type.  The \verb'type' is a
handle, or a pointer to an opaque object.  The handle itself must not be
\verb'NULL' on input, but the content of the handle can be undefined.  On
output, the handle contains a pointer to a newly created type.
The \verb'ctype' is the type in C that will be used to construct the new
GraphBLAS type.  It can be either a built-in C type, or defined by a
\verb'typedef'.
The second parameter should be passed as \verb'sizeof(ctype)'.  The only
requirement on the C type is that \verb'sizeof(ctype)' is valid in C, and
that the type reside in a contiguous block of memory so that it can be moved
with \verb'memcpy'.  For example, to create a user-defined type called
\verb'Complex' for double-precision complex values using the C11
\verb'double complex' type, the following can be used.  A complete example can
be found in the \verb'usercomplex.c' and \verb'usercomplex.h' files in the
\verb'Demo' folder.

    {\footnotesize
    \begin{verbatim}
    #include <math.h>
    #include <complex.h>
    GrB_Type Complex ;
    GrB_Type_new (&Complex, sizeof (double complex)) ;    \end{verbatim} }

To demonstrate the flexibility of the \verb'GrB_Type', consider a ``scalar''
consisting of 4-by-4 floating-point matrix and a string.  This type might be
useful for the 4-by-4 translation/rotation/scaling matrices that arise in
computer graphics, along with a string containing a description or even a
regular expression that can be parsed and executed in a user-defined operator.
All that is required is a fixed-size type, where \verb'sizeof(ctype)' is
a constant.

    {\footnotesize
    \begin{verbatim}
    typedef struct
    {
        float stuff [4][4] ;
        char whatstuff [64] ;
    }
    wildtype ;
    GrB_Type WildType ;
    GrB_Type_new (&WildType, sizeof (wildtype)) ; \end{verbatim} }

With this type a sparse matrix can be created in which each entry consists of a
4-by-4 dense matrix \verb'stuff' and a 64-character string \verb'whatstuff'.
GraphBLAS treats this 4-by-4 as a ``scalar.'' Any GraphBLAS method or operation
that simply moves data can be used with this type without any further
information from the user application.  For example, entries of this type can
be assigned to and extracted from a matrix or vector, and matrices containing
this type can be transposed.  A working example (\verb'wildtype.c'
in the \verb'Demo' folder) creates matrices and multiplies them with
a user-defined semiring with this type.

Performing arithmetic on matrices and vectors with user-defined types requires
operators to be defined.  Refer to Section~\ref{user} for more details on these
example user-defined types.

User defined types created by \verb'GrB_Type_new' will not work with
the JIT; use \verb'GxB_Type_new' instead.

%-------------------------------------------------------------------------------
\subsubsection{{\sf GxB\_Type\_new:} create a user-defined type (with name and definition)}
%-------------------------------------------------------------------------------
\label{type_new_named}

\begin{mdframed}[userdefinedwidth=6in]
{\footnotesize
\begin{verbatim}
GrB_Info GxB_Type_new           // create a new named GraphBLAS type
(
    GrB_Type *type,             // handle of user type to create
    size_t sizeof_ctype,        // size = sizeof (ctype) of the C type
    const char *type_name,      // name of the type (max 128 characters)
    const char *type_defn       // typedef for the type (no max length)
) ;
\end{verbatim}
}\end{mdframed}

\verb'GxB_Type_new' creates a type with a name and definition that are known to
GraphBLAS, as strings.  The \verb'type_name' is any valid string (max length of 128
characters, including the required null-terminating character) that may
appear as the name of a C type created by a C \verb'typedef' statement.  It must
not contain any white-space characters.  For example, to create a type of size
16*4+1 = 65 bytes, with a 4-by-4 dense float array and a 32-bit integer:

    {\footnotesize
    \begin{verbatim}
    typedef struct { float x [4][4] ; int color ; } myquaternion ;
    GrB_Type MyQtype ;
    GxB_Type_new (&MyQtype, sizeof (myquaternion), "myquaternion",
        "typedef struct { float x [4][4] ; int color ; } myquaternion ;") ; \end{verbatim}}

The \verb'type_name' and \verb'type_defn' are both null-terminated strings.
The two strings are optional, but are
required to enable the JIT compilation of kernels that use this type.
At most \verb'GxB_MAX_NAME_LEN' (128) characters are accessed in \verb'type_name';
characters beyond that limit are silently ignored.

If the \verb'sizeof_ctype' is zero, and the strings are valid, a
JIT kernel is compiled just to determine the size of the type.  This is
feature useful for interfaces in languages other than C, which could create
valid strings for C types but would not have a reliable way to determine the
size of the type.

The above example is identical to the following usage, except that
\verb'GrB_Type_new' requires \verb'sizeof_ctype' to be nonzero, and equal
to the size of the C type.

    {\footnotesize
    \begin{verbatim}
    typedef struct { float x [4][4] ; int color ; } myquaternion ;
    GrB_Type MyQtype ;
    GxB_Type_new (&MyQtype, sizeof (myquaternion)) ;
    GrB_set (MyQtype, "myquaternion", GxB_JIT_C_NAME) ;
    GrB_set (MyQtype, "typedef struct { float x [4][4] ; int color ; } myquaternion ;"
        GxB_JIT_C_DEFINITION) ; \end{verbatim}}

%-------------------------------------------------------------------------------
\subsubsection{{\sf GrB\_Type\_wait:} wait for a type}
%-------------------------------------------------------------------------------
\label{type_wait}

\begin{mdframed}[userdefinedwidth=6in]
{\footnotesize
\begin{verbatim}
GrB_Info GrB_wait               // wait for a user-defined type
(
    GrB_Type type,              // type to wait for
    int mode                    // GrB_COMPLETE or GrB_MATERIALIZE
) ;
\end{verbatim}
}\end{mdframed}

After creating a user-defined type, a GraphBLAS library may choose to exploit
non-blocking mode to delay its creation.  Currently, SuiteSparse:GraphBLAS
currently does nothing except to ensure that \verb'type' is valid.

\newpage
%-------------------------------------------------------------------------------
\subsubsection{{\sf GxB\_Type\_from\_name:} return the type from its name}
%-------------------------------------------------------------------------------
\label{type_from_name}

\begin{mdframed}[userdefinedwidth=6in]
{\footnotesize
\begin{verbatim}
GrB_Info GxB_Type_from_name     // return the built-in GrB_Type from a name
(
    GrB_Type *type,             // built-in type, or NULL if user-defined
    const char *type_name       // array of size at least GxB_MAX_NAME_LEN
) ;
\end{verbatim}
}\end{mdframed}

Returns the built-in type from the corresponding name of the type.  The following
examples both return \verb'type' as \verb'GrB_BOOL'.

{\footnotesize
\begin{verbatim}
    GxB_Type_from_name (&type, "bool") ;
    GxB_Type_from_name (&type, "GrB_BOOL") ; \end{verbatim} }

If the name is from a user-defined type, the \verb'type' is returned as
\verb'NULL'.  This is not an error condition.  The user application must itself
do this translation since GraphBLAS does not keep a registry of all
user-defined types.

With this function, a user application can manage the translation for
both built-in types and its own user-defined types, as in the following
example.

{\footnotesize
\begin{verbatim}
    typedef struct { double x ; char stuff [16] ; } myfirsttype ;
    typedef struct { float z [4][4] ; int color ; } myquaternion ;
    GrB_Type MyType1, MyQType ;
    GxB_Type_new (&MyType1, sizeof (myfirsttype), "myfirsttype",
        "typedef struct { double x ; char stuff [16] ; } myfirsttype ;") ;
    GxB_Type_new (&MyQType, sizeof (myquaternion), "myquaternion",
        "typedef struct { float z [4][4] ; int color ; } myquaternion ;") ;

    GrB_Matrix A ;
    // ... create a matrix A of some built-in or user-defined type

    // later on, to query the type of A:
    size_t typesize ;
    GrB_Scalar_new (s, GrB_UINT64) ;
    GrB_get (type, s, GrB_SIZE) ;
    GrB_Scalar_extractElement (&typesize, GrB_UINT64) ;
    GrB_Type atype ;
    char atype_name [GxB_MAX_NAME_LEN] ;
    GrB_get (A, atype_name, GrB_EL_TYPE_STRING) ;
    GxB_Type_from_name (&atype, atype_name) ;
    if (atype == NULL)
    {
        // This is not yet an error.  It means that A has a user-defined type.
        if ((strcmp (atype_name, "myfirsttype")) == 0) atype = MyType1 ;
        else if ((strcmp (atype_name, "myquaternion")) == 0) atype = MyQType ;
        else { ... this is now an error ... the type of A is unknown.  }
    }\end{verbatim} }

%-------------------------------------------------------------------------------
\subsubsection{{\sf GrB\_Type\_free:} free a user-defined type}
%-------------------------------------------------------------------------------
\label{type_free}

\begin{mdframed}[userdefinedwidth=6in]
{\footnotesize
\begin{verbatim}
GrB_Info GrB_free               // free a user-defined type
(
    GrB_Type *type              // handle of user-defined type to free
) ;
\end{verbatim}
}\end{mdframed}

\verb'GrB_Type_free' frees a user-defined type.
Either usage:

    {\small
    \begin{verbatim}
    GrB_Type_free (&type) ;
    GrB_free (&type) ; \end{verbatim}}

\noindent
frees the user-defined \verb'type' and
sets \verb'type' to \verb'NULL'.
It safely does nothing if passed a \verb'NULL'
handle, or if \verb'type == NULL' on input.

It is safe to attempt to free a built-in type.  SuiteSparse:GraphBLAS silently
ignores the request and returns \verb'GrB_SUCCESS'.  A user-defined type should
not be freed until all operations using the type are completed.
SuiteSparse:GraphBLAS attempts to detect this condition but it must query a
freed object in its attempt.  This is hazardous and not recommended.
Operations on such objects whose type has been freed leads to undefined
behavior.

It is safe to first free a type, and then a matrix of that type, but after the
type is freed the matrix can no longer be used.  The only safe thing that can
be done with such a matrix is to free it.

The function signature of \verb'GrB_Type_free' uses the generic name
\verb'GrB_free', which can free any GraphBLAS object. See Section~\ref{free}
details.  GraphBLAS includes many such generic functions.  When describing a
specific variation, a function is described with its specific name in this User
Guide (such as \verb'GrB_Type_free').  When discussing features applicable to
all specific forms, the generic name is used instead (such as \verb'GrB_free').


