
%-------------------------------------------------------------------------------
\subsection{Global Options ({\sf GrB\_Global})}
\label{get_set_global}
%-------------------------------------------------------------------------------

A single object \verb'GrB_GLOBAL' whose type is \verb'GrB_Global' is used to
denote global settings to read or modify.  To use it with \verb'GrB_get' and
\verb'GrB_set', pass in \verb'GrB_GLOBAL' as the first parameter.

\begin{mdframed}[userdefinedwidth=6in]
{\footnotesize
\begin{verbatim}
GrB_Info GrB_get (GrB_Global g, GrB_Scalar value, int f) ;
GrB_Info GrB_get (GrB_Global g, char *     value, int f) ;
GrB_Info GrB_get (GrB_Global g, int32_t *  value, int f) ;
GrB_Info GrB_get (GrB_Global g, size_t *   value, int f) ;
GrB_Info GrB_get (GrB_Global g, void *     value, int f) ;

GrB_Info GrB_set (GrB_Global g, GrB_Scalar value, int f) ;
GrB_Info GrB_set (GrB_Global g, char *     value, int f) ;
GrB_Info GrB_set (GrB_Global g, int32_t    value, int f) ;
GrB_Info GrB_set (GrB_Global g, void *     value, int f, size_t s) ;
\end{verbatim}
}\end{mdframed}


\noindent
{\small
\begin{tabular}{|l|l|l|p{2.5in}|}
\hline
\verb'int field'                    & R/W  & C type        & description \\
\hline
\verb'GrB_LIBRARY_VER_MAJOR'        & R    & \verb'int32_t'& major version of the library \\
\verb'GrB_LIBRARY_VER_MINOR'        & R    & \verb'int32_t'& minor version of the library \\
\verb'GrB_LIBRARY_VER_PATCH'        & R    & \verb'int32_t'& patch version of the library \\
\verb'GrB_API_VER_MAJOR'            & R    & \verb'int32_t'& major version of the API \\
\verb'GrB_API_VER_MINOR'            & R    & \verb'int32_t'& major version of the API \\
\verb'GrB_API_VER_PATCH'            & R    & \verb'int32_t'& major version of the API \\
\verb'GrB_BLOCKING_MODE'            & R    & \verb'int32_t'& blocking mode (\verb'GrB_BLOCKING' \newline
                                                                or \verb'GrB_NONBLOCKING') \\
\verb'GxB_LIBRARY_OPENMP'           & R    & \verb'int32_t'& if OpenMP is in use (true/false) \\
\verb'GrB_STORAGE_ORIENTATION_HINT' & R/W  & \verb'int32_t'& see \verb'GrB_Orientation': default
                                                            format for matrices. \\
%                                                           Matrices are held by row, unless this
%                                                           value is set to \verb'GrB_COLMAJOR'. \\
\verb'GxB_NTHREADS'                 & R/W  & \verb'int32_t'& number of OpenMP threads used. \newline
                                                            See Section~\ref{omp_parallelism}. \\
% \verb'GxB_GPU_ID'                 & R/W  & \verb'int32_t'& which GPU to use \\
\verb'GxB_BURBLE'                   & R/W  & \verb'int32_t'& diagnostic output (true/false). \newline
                                                                See Section~\ref{diag}. \\
\verb'GxB_PRINT_1BASED'             & R/W  & \verb'int32_t'& matrices printed as 1-based or 0-based  \\
\verb'GxB_INCLUDE_READONLY_STATISTICS' &R/W& \verb'int32_t'& include read-only memory in statistics \\
\verb'GxB_JIT_C_CONTROL'            & R/W  & \verb'int32_t'& see Section~\ref{jit} \\
\verb'GxB_JIT_USE_CMAKE'            & R/W  & \verb'int32_t'& " \\
\verb'GxB_ROWINDEX_INTEGER_HINT'    & R/W  & \verb'int32_t'& hint for row indices (32 or 64) \\
\verb'GxB_COLINDEX_INTEGER_HINT'    & R/W  & \verb'int32_t'& hint for column indices (32 or 64) \\
\verb'GxB_OFFSET_INTEGER_HINT'      & R/W  & \verb'int32_t'& hint for offsets (32 or 64) \\
\hline
\verb'GxB_HYPER_SWITCH'             & R/W  & \verb'double' & global hypersparsity control. \newline
                                                                See Section~\ref{hypersparse}. \\
\verb'GxB_HYPER_HASH'               & R/W  & \verb'int64_t' & global hypersparsity (hyper-hash)
                                                                control \\
\verb'GxB_CHUNK'                    & R/W  & \verb'double' & global chunk size for parallel task creation.
                                                                See Section~\ref{omp_parallelism}. \\
\hline
\verb'GrB_NAME'                     & R    & \verb'char *' & name of the library \newline
                                                                (\verb'"SuiteSparse:GraphBLAS"') \\
\verb'GxB_LIBRARY_DATE'             & R    & \verb'char *' & library release date \\
\verb'GxB_LIBRARY_ABOUT'            & R    & \verb'char *' & details about the library \\
\verb'GxB_LIBRARY_LICENSE'          & R    & \verb'char *' & license of the library \\
\verb'GxB_LIBRARY_COMPILE_DATE'     & R    & \verb'char *' & date the library was compiled \\
\verb'GxB_LIBRARY_COMPILE_TIME'     & R    & \verb'char *' & time the library was compiled \\
\verb'GxB_LIBRARY_URL'              & R    & \verb'char *' & URL for the library \\
\verb'GxB_API_DATE'                 & R    & \verb'char *' & C API release date \\
\verb'GxB_API_ABOUT'                & R    & \verb'char *' & about the C API \\
\verb'GxB_API_URL'                  & R    & \verb'char *' & URL for the C API \\
\verb'GxB_COMPILER_NAME'            & R    & \verb'char *' & name of the compiler used to compile the library \\
\hline
\end{tabular}
}
\vspace{0.05in}

\vspace{0.1in}
\noindent
{\small
\begin{tabular}{|l|l|l|p{2.5in}|}
\hline
\verb'int field'                    & R/W  & C type        & description \\
\hline
\verb'GxB_JIT_C_COMPILER_NAME'      & R/W  & \verb'char *' & See Section~\ref{jit} \\
\verb'GxB_JIT_C_COMPILER_FLAGS'     & R/W  & \verb'char *' & " \\
\verb'GxB_JIT_C_LINKER_FLAGS'       & R/W  & \verb'char *' & " \\
\verb'GxB_JIT_C_LIBRARIES'          & R/W  & \verb'char *' & " \\
\verb'GxB_JIT_C_CMAKE_LIBS'         & R/W  & \verb'char *' & " \\
\verb'GxB_JIT_C_PREFACE'            & R/W  & \verb'char *' & " \\
\verb'GxB_JIT_ERROR_LOG'            & R/W  & \verb'char *' & " \\
\verb'GxB_JIT_CACHE_PATH'           & R/W  & \verb'char *' & " \\
\hline
\verb'GxB_BITMAP_SWITCH'            & R/W  & \verb'void *' & \verb'double' array of size \newline
                                                                \verb'GxB_NBITMAP_SWITCH'.  \newline
                                                                See Section~\ref{bitmap_switch}. \\
\verb'GxB_COMPILER_VERSION'         & R    & \verb'void *' & \verb'int32_t' array of size 3.
                                        The version of the compiler used to
                                        compile the library. \\
\verb'GxB_PRINTF'                   & W    & \verb'void *' & pointer to \verb'printf' function for diagnostic output.
                                                                See Section~\ref{diag}. \\
\verb'GxB_FLUSH'                    & W    & \verb'void *' & pointer to \verb'flush' function for diagnostic output.
                                                                See Section~\ref{diag}. \\
\verb'GxB_MALLOC_FUNCTION'  & R    & \verb'void *' & malloc function \\
\verb'GxB_CALLOC_FUNCTION'  & R    & \verb'void *' & calloc function \\
\verb'GxB_REALLOC_FUNCTION' & R    & \verb'void *' & realloc function \\
\verb'GxB_FREE_FUNCTION'    & R    & \verb'void *' & free function \\
\hline
\end{tabular}
}

%-------------------------------------------------------------------------------
\subsubsection{Global diagnostic settings}
\label{diag}
%-------------------------------------------------------------------------------

\verb'GrB_set (GrB_GLOBAL, ..., GxB_BURBLE)' controls the burble setting.  It can also be
controlled via \verb'GrB.burble(b)' in the MATLAB/Octave interface.

{\footnotesize
\begin{verbatim}
     GrB_set (GrB_GLOBAL, true,  GxB_BURBLE) ;  // enable burble
     GrB_set (GrB_GLOBAL, false, GxB_BURBLE) ;  // disable burble \end{verbatim}}

If enabled, SuiteSparse:GraphBLAS reports which internal kernels it uses, and
how much time is spent.  If you see the word \verb'generic', it means that
SuiteSparse:GraphBLAS was unable to use its JIT kernels, or its faster kernels
in \verb'Source/FactoryKernels', but used a generic kernel that relies on
function pointers.  This is done for user-defined types and operators when they
cannot be used in the JIT, and when typecasting is performed.  Generic kernels
are typically slower than the JIT kernels or kernels in
\verb'Source/FactoryKernels'.

If you see a lot of \verb'wait' statements, it may mean that a lot of time is
spent finishing a matrix or vector.  This may be the result of an inefficient
use of the \verb'setElement' and \verb'assign' methods.  If this occurs you
might try changing the sparsity format of a vector or matrix to
\verb'GxB_BITMAP', assuming there's enough space for it.

The following setting allows the user application to change the
function used to print diagnostic output:

{\small
\begin{verbatim}
    GrB_set (GrB_GLOBAL, (void *) printf, GxB_PRINTF, sizeof (void *)) ; \end{verbatim} }

This also controls the output of the
\verb'GxB_*print' functions.  By default this parameter is \verb'NULL', in
which case the C11 \verb'printf' function is used.  The parameter is a
function pointer with the same signature as the C11 \verb'printf'
function.  The MATLAB/Octave interface to GraphBLAS sets it to \verb'mexPrintf'
so that GraphBLAS can print to the MATLAB/Octave Command Window.

After each call to the \verb'printf' function, an optional
\verb'flush' function is called, which is \verb'NULL' by default.  If
\verb'NULL', the function is not used.  This can be changed with:

{\small
\begin{verbatim}
    GrB_set (GrB_GLOBAL, (void *) flush, GxB_FLUSH, sizeof (void *)) ; \end{verbatim} }

The \verb'flush' function takes no
arguments, and returns an \verb'int' which is 0 if successful, or any nonzero
value on failure (the same output as the C11 \verb'fflush' function,
except that \verb'flush' has no inputs).

%-------------------------------------------------------------------------------
\subsubsection{OpenMP parallelism}
%-------------------------------------------------------------------------------
\label{omp_parallelism}

SuiteSparse:GraphBLAS is a parallel library, based on OpenMP.  By
default, all GraphBLAS operations will use up to the maximum number of threads
specified by the \verb'omp_get_max_threads' OpenMP function.  For small
problems, GraphBLAS may choose to use fewer threads, using two parameters: the
maximum number of threads to use (which may differ from the
\verb'omp_get_max_threads' value), and a parameter called the \verb'chunk'.
Suppose \verb'work' is a measure of the work an operation needs to perform (say
the number of entries in the two input matrices for \verb'GrB_eWiseAdd').  No
more than \verb'floor(work/chunk)' threads will be used (or one thread if the
ratio is less than 1).

\verb'GxB_NTHREADS' controls how many threads a method uses.
    By default (if set to zero, or \verb'GrB_DEFAULT'), all available threads
    are used.  The maximum available threads is controlled by the global
    setting, which is \verb'omp_get_max_threads ( )' by default.  If set to
    some positive integer \verb'nthreads' less than this maximum, at most
    \verb'nthreads' threads will be used.

\verb'GxB_CHUNK' is a \verb'double' value that controls how many threads
    a method uses for small problems.
The default \verb'chunk' value is 65,536, but this may change in future
versions, or it may be modified when GraphBLAS is installed on a particular
machine.

Both parameters can be set in two ways:

\begin{itemize}

\item Globally:  If the following methods are used, then all subsequent
GraphBLAS operations will use these settings.  Note the typecast,
\verb'(double)' \verb'chunk'.  This is necessary if a literal constant such as
\verb'20000' is passed as this argument.  The type of the constant must be
\verb'double'.

    {\footnotesize
    \begin{verbatim}
    int32_t nthreads_max = 40 ;
    GrB_set (GrB_GLOBAL, nthreads_max, GxB_NTHREADS) ;
    GrB_Scalar_new (&s, GrB_FP64) ;
    GrB_Scalar_setElement (s, (double) 20000) ;
    GrB_set (GrB_GLOBAL, s, GxB_CHUNK) ; \end{verbatim} }

\item Context: this object can be used to choose a different number of
threads used in calls to GraphBLAS from different user threads, exploiting
nested parallelism.  Refer to Section~\ref{context}.  If a thread has engaged a
context object, it ignores the global settings for \verb'GxB_NTHREADS' and
\verb'GxB_CHUNK', and uses the settings in its own context instead.

\end{itemize}

The smaller of \verb'nthreads_max' and \verb'floor(work/chunk)' is used for any
given GraphBLAS operation, except that a single thread is used if this value is
zero or less.

If either parameter is set to \verb'GrB_DEFAULT', then default values are used.
The default for \verb'nthreads_max' is the return value from
\verb'omp_get_max_threads', and the default chunk size is currently 65,536.

If a descriptor value for either parameter is left at its default, or set to
\verb'GrB_DEFAULT', then the global setting is used.  This global setting may
have been modified from its default, and this modified value will be used.

For example, suppose \verb'omp_get_max_threads' reports 8 threads.  If \newline
\verb'GrB_set (GrB_GLOBAL, 4, GxB_NTHREADS)' is used, then the global setting is four
threads, not eight.

GraphBLAS may be compiled without OpenMP, by setting \verb'-DNOPENMP=1'.
The library will be thread-safe, with one exception.  \verb'GrB_wait' is
intended to provide thread-safety by flushing the cache of one user thread
so the object can be safely read by another thread.  This is accomplished
with \verb'pragma omp flush', but if OpenMP is not available, this does
nothing.  If OpenMP is not available or \verb'-DNOPEMP=1' is used, then
user applications need to ensure their own thread safety when one user thread
computes a result that is then read by another thread.

You can query GraphBLAS at run-time to ask if it was compiled with OpenMP:

{\small
\begin{verbatim}
    bool have_openmp ;
    GrB_get (GrB_GLOBAL, &have_openmp, GxB_LIBRARY_OPENMP) ;
    if (!have_openmp) printf ("GraphBLAS not compiled with OpenMP\n") ; \end{verbatim}}

Compiling GraphBLAS without OpenMP is not recommended for installation in a
package manager (Linux, conda-forge, spack, brew, vcpkg, etc).

%-------------------------------------------------------------------------------
\subsubsection{Other global options}
%-------------------------------------------------------------------------------

\verb'GrB_BLOCKING_MODE' can only be queried by \verb'GrB_get'; it cannot be
modified by \verb'GrB_set'.  The mode is the value passed to \verb'GrB_init'
(blocking or non-blocking).

All threads in the same user application share the same global options,
including hypersparsity, bitmap options, and CSR/CSC format determined by
\verb'GrB_set', and the blocking mode determined by \verb'GrB_init'.
Specific format and hypersparsity parameters of each matrix are specific to
that matrix and can be independently changed.

The \verb'GrB_LIBRARY_*' and \verb'GxB_LIBRARY_*' options can be used to query
the current implementation of SuiteSparse:GraphBLAS.  The \verb'GrB_API_*' and
\verb'GxB_API_*' options can be used to query the current GraphBLAS C API
Specification.

