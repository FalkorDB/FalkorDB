
\newpage
%-------------------------------------------------------------------------------
\section{Compiling and Installing SuiteSparse:GraphBLAS}
%-------------------------------------------------------------------------------
\label{sec:install}

%----------------------------------------
\subsection{Quick Start}
%----------------------------------------

GraphBLAS requires \verb'cmake' version 3.20 or later.
It optionally can use OpenMP for best performance.
For OpenMP on the Mac, see Section~\ref{mac_openmp}.
Without OpenMP, GraphBLAS will be significantly slower since it is a highly
parallel package.

A \verb'cmake' build system is available for Linux, Mac, and Windows.
For Linux or Mac, a simple Makefile wrapper is available that accesses this
cmake build system.  Simply do:

    {\small
    \begin{verbatim}
    make
    sudo make install \end{verbatim} }

For Windows, open CMake and use the provided \verb'CMakeLists.txt' file
to build GraphBLAS in the \verb'GraphBLAS/build' folder.

Next, try the demos with \verb'make demos'.
The output of the demos will be compared with expected
output files in \verb'Demo/Output'.\footnote{
NOTE:
DO NOT publish benchmarks of these demos, and do not link against the
demo library in any user application.  These codes are sometimes slow,
and are meant as simple illustrations only, not for performance.  The fastest
methods are in LAGraph, not in GraphBLAS/Demo.  Benchmark LAGraph
instead.  Eventually, all GraphBLAS/Demos methods will be removed, and LAGraph
will serve all uses: for illustration, benchmarking, and production uses.}

%----------------------------------------
\subsection{Requirements}
%----------------------------------------

GraphBLAS requires \verb'cmake' version 3.20 or later.
It optionally can use OpenMP for best performance.
It requires the lz4 and zstd compression packages, but they are bundled
with the GraphBLAS distribution and compiled into the GraphBLAS library
by the cmake script.

Using the GraphBLAS JIT has many requirements, but it greatly increases
performance, particularly with user-defined types and operators, and when
typecasting is done with built-in types and operators.

The cmake build script sets up the JIT to use the compiler and compiler
flags used to build GraphBLAS itself.  If the target environment does not
have a compiler (the Apple iPad for example), then the JIT will not work.

The JIT requires access to a cache folder, which should be preserved
even after a program that uses GraphBLAS finishes, so that JIT kernels do not
have to continually be recompiled.  By default the folder is
\verb'~/.SuiteSparse/GrBX.Y.Z' on Linux/Mac, where \verb'X.Y.Z' is the version
of GraphBLAS you have.  On Windows, the location is determined by your
\verb'LOCALAPPDATA' environment variable, where the folder is
\verb'SuiteSparse/GrBX.Y.Z' inside that location.  File locking is required,
so if the file system does not support file locking, the JIT will not work and
performance will suffer (the \verb'make demos' will fail since it tests this
condition).  If this occurs, try making your \verb'SuiteSparse' cache folder
a link to another a location on another file system that supports file locking,
or set an environment variable, \verb'GRAPHBLAS_CACHE_PATH', to point to your
a cache folder location on a file system that supports file locking.

%----------------------------------------
\subsection{Quick Start for MATLAB/Octave}
%----------------------------------------

As of GraphBLAS 9.2.0, a new and simpler method for compiling GraphBLAS and its
MATLAB interface has been added.  In the MATLAB/Octave Command Window, simply
type:

    {\small
    \begin{verbatim}
    cd GraphBLAS/GraphBLAS
    graphblas_install \end{verbatim} }

This will use \verb'cmake' to compile the GraphBLAS library.  You can skip the
details in the remainder of this section.  Next, add your
\verb'GraphBLAS/GraphBLAS' folder to your path, by editting your
\verb'startup.m' script (usually in your \verb'Documents/MATLAB' folder).
Add this line:

    {\small
    \begin{verbatim}
    addpath ('/home/me/GraphBLAS/GraphBLAS') ; \end{verbatim} }

\noindent
where \verb'/home/me/GraphBLAS' is the top-level folder containing your
copy of GraphBLAS.

The \verb'graphblas_install' MATLAB script may fail to run \verb'cmake'.
If it does, it will print the following workaround, where the commands it
tells you to use will differ depending on the platform:

    {\small
    \begin{verbatim}
    Building GraphBLAS with cmake failed.  Try this outside of MATLAB:

       cd /home/me/GraphBLAS/GraphBLAS/build
       cmake  ..
       cmake --build . --config Release -j40

    Then do this inside MATLAB:

        cd /home/me/GraphBLAS/GraphBLAS/@GrB/private
        gbmake \end{verbatim} }

\noindent
where \verb'/home/me/GraphBLAS' is your copy of GraphBLAS.

%----------------------------------------
\subsection{More details}
%----------------------------------------

%----------------------------------------
\subsubsection{On Linux and Mac}
%----------------------------------------

GraphBLAS makes extensive use of features in the C11 standard, and thus a
C compiler supporting this version of the C standard is required to use
all features of GraphBLAS.

{\bf Any version of the Intel \verb'icx' compiler is highly recommended.} In
most cases, the Intel \verb'icx' and the Intel OpenMP library (\verb'libiomp')
result in the best performance.  The \verb'gcc' and the GNU OpenMP library
(\verb'libgomp') generally gives good performance: typically on par with icx
but in a few special cases significantly slower.  The Intel \verb'icc' compiler
is not recommended; it results in poor performance for
\verb'#pragma omp atomic'.

If you are using a C compiler that does not support the C11 standard, such as
\verb'cl' in Microsoft Visual Studio, then the \verb'_Generic' keyword is not
available.  SuiteSparse:GraphBLAS will still compile, but you will not have
access to polymorphic functions such as \verb'GrB_assign'.  You will need to
use the non-polymorphic functions instead.

To compile SuiteSparse:GraphBLAS, simply type \verb'make' in the main GraphBLAS
folder, which compiles the library with your default system compiler.  This
compile GraphBLAS using 8 threads, which will take a long time.  To compile with
more threads (40, for this example), use:

    {\small
    \begin{verbatim}
    make JOBS=40 \end{verbatim} }

To use a non-default compiler with 4 threads:

    {\small
    \begin{verbatim}
    make CC=icx CXX=icpx JOBS=4 \end{verbatim} }

GraphBLAS v6.1.3 and later use the \verb'cpu_features' package by Google to
determine if the target architecture supports AVX2 and/or AVX512F (on Intel
x86\_64 architectures only).  In case you have build issues with this package,
you can compile without it (and then AVX2 and AVX512F acceleration will not
be used):

    {\small
    \begin{verbatim}
    make CMAKE_OPTIONS='-DGBNCPUFEAT=1'  \end{verbatim} }

Without \verb'cpu_features', it is still possible to enable AVX2 and AVX512F.
Rather than relying on run-time tests, you can use these flags to enable
both AVX2 and AVX512F, without relying on \verb'cpu_features':

    {\small
    \begin{verbatim}
    make CMAKE_OPTIONS='-DGBNCPUFEAT=1 -DGBAVX2=1 -DGBAVX512F=1' \end{verbatim} }

To use multiple options, separate them by a space.  For example, to build
just the library but not \verb'cpu_features', and to enable
AVX2 but not AVX512F, and use 40 threads to compile:

    {\small
    \begin{verbatim}
    make CMAKE_OPTIONS='-DGBNCPUFEAT=1 -DGBAVX2=1' JOBS=40 \end{verbatim} }

After compiling the library, you can compile the demos with
\verb'make all' and then \verb'make demos' while in the top-level
GraphBLAS folder.

If \verb'cmake' or \verb'make' fail, it might be that your default compiler
does not support C11.  Try another compiler.  For example, try one of
these options.  Go into the \verb'build' directory and type one of these:

    {\small
    \begin{verbatim}
    CC=gcc cmake ..
    CC=gcc-11 cmake ..
    CC=xlc cmake ..
    CC=icx cmake ..  \end{verbatim} }

You can also do the following in the top-level GraphBLAS folder instead:

    {\small
    \begin{verbatim}
    CC=gcc make
    CC=gcc-11 make
    CC=xlc make
    CC=icx make \end{verbatim} }

For faster compilation, you can specify a parallel make.  For example,
to use 32 parallel jobs and the \verb'gcc' compiler, do the following:

    {\small
    \begin{verbatim}
    JOBS=32 CC=gcc make \end{verbatim} }

%----------------------------------------
\subsubsection{On the Mac (Intel or ARM)}
%----------------------------------------

GraphBLAS requires \verb'cmake' v3.20 or later, and it optionally uses
\verb'make' to simplify the use of the cmake build system.  It also needs
other Apple Command Line Tools from Xcode.
First install Xcode (see \url{https://developer.apple.com/xcode}),
and then install the command line tools for Xcode:

    {\small
    \begin{verbatim}
    xcode-select --install \end{verbatim} }

Next, install brew, at \url{https://brew.sh}, or \verb'spack'.
Use brew or spack to install cmake version 3.20 or later.

%----------------------------------------
\subsubsection{On the Intel-based Mac}
%----------------------------------------

If you have the Intel compiler and its OpenMP library, then use the following
in the top-level \verb'GraphBLAS' folder.  The Intel OpenMP library
will be found automatically:

    {\small
    \begin{verbatim}
    make CC=icc CXX=icpc \end{verbatim} }

%----------------------------------------
\subsubsection{MATLAB on the Mac (Apple Silicon based)}
%----------------------------------------

MATLAB on the Apple-Silicon-based Mac is now a native ARM64 application (as of
R2023b).  GraphBLAS is not supported for earlier versions of MATLAB on Apple
Silicon.

Note that GraphBLAS must use the same OpenMP library as MATLAB.  This is
handled by the \verb'graphblas_install.m' script.

%----------------------------------------
\subsubsection{On Microsoft Windows}
\label{sec:windows}
%----------------------------------------

SuiteSparse:GraphBLAS can be compiled by the Microsoft C compiler (\verb'cl')
using Microsoft Visual Studio.  However, that compiler is not C11 compliant. As
a result, GraphBLAS on Windows will have a few minor limitations.

\begin{itemize}
\item The MS Visual Studio compiler does not support the \verb'_Generic'
keyword, required for the polymorphic GraphBLAS functions.  So for example, you
will need to use \verb'GrB_Matrix_free' instead of just \verb'GrB_free'.

\item Variable-length arrays are not supported, so user-defined
types are limited to 1024 bytes in size.  This can be changed by editing
\verb'GB_VLA_MAXSIZE' in \verb'Source/GB_compiler.h', and recompiling
SuiteSparse:GraphBLAS.

\item AVX acceleration is not enabled.

\item You must compile with 64-bit computing enabled (x64).  Otherwise, a
compiler error will occur (\verb'InterlockedExchange64', \verb'InterlockdOr64'
and other methods will not be found).

\end{itemize}

If you use a recent \verb'gcc' or \verb'icx' compiler on Windows other than the
Microsoft Compiler (\verb'cl'), these limitations can be avoided.

The following instructions apply to Windows 10, CMake 3.16, and
Visual Studio 2019, but may work for earlier versions.

\begin{enumerate}

\item Install CMake 3.16 or later, if not already installed.
    See \url{https://cmake.org/} for details.

\item Install Microsoft Visual Studio, if not already installed.
    See \url{https://visualstudio.microsoft.com/} for details.
    Version 2019 is preferred, but earlier versions may also work.

\item Open a terminal window and type this in the
    \verb'GraphBLAS/build' folder:

    \vspace{-0.1in}
    {\small
    \begin{verbatim}
    cmake ..  \end{verbatim} }
    \vspace{-0.1in}

    Alternatively, use the \verb'cmake' gui program to configure
    the cmake build system for GraphBLAS.

\item The \verb'cmake' command generates many files in
    \verb'GraphBLAS/build', and the file \verb'graphblas.sln' in
    particular.  Open the generated \verb'graphblas.sln' file in Visual Studio.

\item Optionally: right-click \verb'graphblas' in the left panel (Solution
    Explorer) and select properties; then navigate to \verb'Configuration'
    \verb'Properties', \verb'C/C++', \verb'General' and change the parameter
    \verb'Multiprocessor Compilation' to \verb'Yes (/MP)'.  Click \verb'OK'.
    This will significantly speed up the compilation of GraphBLAS.

\item Select the \verb'Build' menu item at the top of the window and
    select \verb'Build Solution'.  This should create a folder called
    \verb'Release' and place the compiled \verb'graphblas.dll',
    \verb'graphblas.lib', and \verb'graphblas.exp' files there.  Please be
    patient; some files may take a while to compile and sometimes may appear to
    be stalled.  Just wait.

\item Alternatively, instead of opening Visual Studio,
    type this command in the terminal window while in the \verb'build' folder:
    {\small
    \begin{verbatim}
    cmake --build . --config Release
    \end{verbatim}}

\item Add the \verb'GraphBLAS/build/Release' folder to the Windows System path:

    \begin{itemize}
    \item Open the \verb'Start Menu' and type \verb'Control Panel'.
    \item Select the \verb'Control Panel' app.
    \item When the app opens, select \verb'System and Security'.
    \item Under \verb'System and Security', select \verb'System'.
    \item From the top left side of the \verb'System' window, select
        \verb'Advanced System Settings'.  You may have to authenticate
        at this step.
        If you cannot authenticate, try setting the User Environment Variables
        instead.
    \item The \verb'Systems Properties' window should appear with the
        \verb'Advanced' tab selected;
        select \verb'Environment Variables'.
    \item The \verb'Environment Variables' window displays 2 sections, one for
        \verb'User' variables and the other for \verb'System' variables.  Under
        the \verb'Systems' variable section, scroll to and select \verb'Path',
        then select \verb'Edit'.   A editor window appears allowing to add,
        modify, delete or re-order the parts of the \verb'Path'.
    \item Add the full path of the \verb'GraphBLAS\build\Release' folder
        (typically starting with \verb'C:\Users\you\'..., where \verb'you' is
        your Windows username) to the \verb'Path'.
        To use the MATLAB interface, add
        the full path of the \verb'GraphBLAS\GraphBLAS\build\Release' folder
        as well.
    \item If the above steps do not work, you can instead copy the
        \verb'graphblas.*' files from \verb'GraphBLAS\build\Release' into any
        existing folder listed in your \verb'Path'.
    \end{itemize}

\item The \verb'GraphBLAS/Include/GraphBLAS.h' file must be included in user
    applications via \verb'#include "GraphBLAS.h"'.  This is already done for
    you in the MATLAB/Octave interface discussed in the next section.

\end{enumerate}

%----------------------------------------
\subsubsection{Mac using clang}
\label{mac_openmp}
%----------------------------------------

To use OpenMP with clang on the Mac, try installing it from the
    \url{https://mac.r-project.org/openmp/} project.
Be sure to check that page for the OpenMP version
that matches your version of Apple Xcode.
For example, if using Xcode 13.3 to 13.4.1, use:

    {\scriptsize
    \begin{verbatim}
    curl -O https://mac.r-project.org/openmp/openmp-13.0.0-darwin21-Release.tar.gz
    sudo tar fvxz openmp-13.0.0-darwin21-Release.tar.gz -C /
    \end{verbatim} }

These commands will install universal binaries (ARM and x86) for
\verb'libomp.dylib', and the following files:

    {\scriptsize
    \begin{verbatim}
    /usr/local/lib/libomp.dylib
    /usr/local/include/ompt.h
    /usr/local/include/omp.h
    /usr/local/include/omp-tools.h
    \end{verbatim} }

Once you do this, the GraphBLAS cmake build system should find this copy of
OpenMP for clang.

%----------------------------------------
\subsubsection{Linking issues after installation}
%----------------------------------------

My Linux distro (Ubuntu 18.04) includes a copy of \verb'libgraphblas.so.1',
which is SuiteSparse:GraphBLAS v1.1.2.  After installing SuiteSparse:GraphBLAS
in \verb'/usr/local/lib' (with \verb'sudo make install'), compiling a simple
stand-alone program links against \verb'libgraphblas.so.1' instead of the
latest version, while at the same time accessing the latest version of the
include file as \verb'/usr/local/include/GraphBLAS.h'.  This command fails:

    {\small
    \begin{verbatim}
    gcc prog.c -lgraphblas \end{verbatim} }

Revising my \verb'LD_LIBRARY_PATH' to put \verb'/usr/local/lib' first in the
library directory order didn't help.  If you encounter this problem, try one of
the following options (all four work for me, and link against the proper
version, \verb'/usr/local/lib/libgraphblas.so.9.2.0' for example):

    {\small
    \begin{verbatim}
    gcc prog.c -l:libgraphblas.so.9
    gcc prog.c -l:libgraphblas.so.9.2.0
    gcc prog.c /usr/local/lib/libgraphblas.so
    gcc prog.c -Wl,-v -L/usr/local/lib -lgraphblas \end{verbatim} }

This \verb'prog.c' test program is a trivial one, which works in v1.0 and
later:

    {\small
    \begin{verbatim}
    #include <GraphBLAS.h>
    int main (void)
    {
        GrB_init (GrB_NONBLOCKING) ;
        GrB_finalize ( ) ;
    } \end{verbatim} }

Compile the program above, then use this command to ensure
\verb'libgraphblas.so.9' appears:

    {\small
    \begin{verbatim}
    ldd a.out \end{verbatim} }

%----------------------------------------
\subsubsection{Running the tests}
%----------------------------------------

To run a short test, type \verb'make demo' at the top-level \verb'GraphBLAS'
folder.  This will run all the demos in \verb'GraphBLAS/Demos'.  MATLAB is not
required.

To perform the extensive tests in the \verb'Test' folder, and the statement
coverage tests in \verb'Tcov', MATLAB R2018a or later is required.  See the
\verb'README.txt' files in those two folders for instructions on how to run the
tests.  The tests in the \verb'Test' folder have been ported to MATLAB on
Linux, MacOS, and Windows.  The \verb'Tcov' tests do not work on Windows.  The
MATLAB interface test (\verb'gbtest') works on all platforms; see the
\verb'GraphBLAS/GraphBLAS' folder for more details.

%----------------------------------------
\subsubsection{Cleaning up}
%----------------------------------------

To remove all compiled files, type \verb'make' \verb'distclean' in the top-level
GraphBLAS folder.

