
\newpage
%===============================================================================
\subsection{GraphBLAS binary operators: {\sf GrB\_BinaryOp}, $z=f(x,y)$} %======
%===============================================================================
\label{binaryop}

A binary operator is a scalar function of the form $z=f(x,y)$.  The types of
$z$, $x$, and $y$ need not be the same.  The built-in binary operators are
listed in the tables below.  The notation $T$ refers to any of the 13
built-in types, but two of those types are SuiteSparse extensions
(\verb'GxB_FC32' and \verb'GxB_FC64').  For those types, the operator name
always starts with \verb'GxB', not \verb'GrB').
The notation $R$ refers to any real type (all but \verb'FC32' and \verb'FC64').

The six \verb'GxB_IS*' comparators and the \verb'GxB_*' logical
operators all return a result one for true and zero for false, in the same
domain $T$ or $R$ as their inputs.  These six comparators are useful
as ``multiply'' operators for creating semirings with non-Boolean monoids.

\vspace{0.2in}
{\footnotesize
\begin{tabular}{|llll|}
\hline
\multicolumn{4}{|c|}{Binary operators for all 13 types} \\
\hline
GraphBLAS name        & types (domains)            & $z=f(x,y)$      & description \\
\hline
% numeric TxT->T
\verb'GrB_FIRST_'$T$  & $T \times T \rightarrow T$ & $z = x$         & first argument \\
\verb'GrB_SECOND_'$T$ & $T \times T \rightarrow T$ & $z = y$         & second argument \\
\verb'GxB_ANY_'$T$    & $T \times T \rightarrow T$ & $z = x$ or $y$  & pick $x$ or $y$ arbitrarily \\
\verb'GrB_ONEB_'$T$   & $T \times T \rightarrow T$ & $z = 1$         & one \\
\verb'GxB_PAIR_'$T$   & $T \times T \rightarrow T$ & $z = 1$         & one (historical) \\
\verb'GrB_PLUS_'$T$   & $T \times T \rightarrow T$ & $z = x+y$       & addition \\
\verb'GrB_MINUS_'$T$  & $T \times T \rightarrow T$ & $z = x-y$       & subtraction \\
\verb'GxB_RMINUS_'$T$ & $T \times T \rightarrow T$ & $z = y-x$       & reverse subtraction \\
\verb'GrB_TIMES_'$T$  & $T \times T \rightarrow T$ & $z = xy$        & multiplication \\
\verb'GrB_DIV_'$T$    & $T \times T \rightarrow T$ & $z = x/y$       & division \\
\verb'GxB_RDIV_'$T$   & $T \times T \rightarrow T$ & $z = y/x$       & reverse division \\
\verb'GxB_POW_'$T$    & $T \times T \rightarrow T$ & $z = x^y$       & power \\
\hline
% TxT->T comparators
\verb'GxB_ISEQ_'$T$   & $T \times T \rightarrow T$ & $z = (x == y)$  & equal \\
\verb'GxB_ISNE_'$T$   & $T \times T \rightarrow T$ & $z = (x \ne y)$ & not equal \\
\hline
\end{tabular}
}
\vspace{0.2in}

The \verb'GxB_POW_*' operators for real types do not return a complex result,
and thus $z = f(x,y) = x^y$ is undefined if $x$ is negative and $y$ is not an
integer.  To compute a complex result, use \verb'GxB_POW_FC32' or
\verb'GxB_POW_FC64'.

Operators that require the domain to be ordered (\verb'MIN', \verb'MAX',
less-than, greater-than, and so on) are not defined for
complex types.  These are listed in the following table:

\vspace{0.2in}
{\footnotesize
\begin{tabular}{|llll|}
\hline
\multicolumn{4}{|c|}{Binary operators for all non-complex types} \\
\hline
GraphBLAS name        & types (domains)            & $z=f(x,y)$      & description \\
\hline
% numeric RxR->R
\verb'GrB_MIN_'$R$    & $R \times R \rightarrow R$ & $z = \min(x,y)$ & minimum \\
\verb'GrB_MAX_'$R$    & $R \times R \rightarrow R$ & $z = \max(x,y)$ & maximum \\
\hline
% RxR->R comparators
\verb'GxB_ISGT_'$R$   & $R \times R \rightarrow R$ & $z = (x >   y)$ & greater than \\
\verb'GxB_ISLT_'$R$   & $R \times R \rightarrow R$ & $z = (x <   y)$ & less than  \\
\verb'GxB_ISGE_'$R$   & $R \times R \rightarrow R$ & $z = (x \ge y)$ & greater than or equal \\
\verb'GxB_ISLE_'$R$   & $R \times R \rightarrow R$ & $z = (x \le y)$ & less than or equal  \\
\hline
% RxR->R logical
\verb'GxB_LOR_'$R$    & $R \times R \rightarrow R$ & $z = (x \ne 0) \vee    (y \ne 0) $ & logical OR \\
\verb'GxB_LAND_'$R$   & $R \times R \rightarrow R$ & $z = (x \ne 0) \wedge  (y \ne 0) $ & logical AND \\
\verb'GxB_LXOR_'$R$   & $R \times R \rightarrow R$ & $z = (x \ne 0) \veebar (y \ne 0) $ & logical XOR \\
\hline
\end{tabular}
}
\vspace{0.2in}

Another set of six kinds of built-in comparators have the form $T
\times T \rightarrow $\verb'bool'.  Note that when $T$ is \verb'bool', the six
operators give the same results as the six \verb'GxB_IS*_BOOL' operators in the
table above.  These six comparators are useful as ``multiply''
operators for creating semirings with Boolean monoids.

\vspace{0.2in}
{\footnotesize
\begin{tabular}{|llll|}
\hline
\multicolumn{4}{|c|}{Binary comparators for all 13 types} \\
\hline
GraphBLAS name        & types (domains)            & $z=f(x,y)$      & description \\
\hline
% 6 TxT -> bool comparators
\verb'GrB_EQ_'$T$     & $T \times T \rightarrow $\verb'bool' & $z = (x == y)$  & equal \\
\verb'GrB_NE_'$T$     & $T \times T \rightarrow $\verb'bool' & $z = (x \ne y)$ & not equal \\
\hline
\multicolumn{4}{ }{\mbox{ }} \\
\hline
\multicolumn{4}{|c|}{Binary comparators for non-complex types} \\
\hline
GraphBLAS name        & types (domains)            & $z=f(x,y)$      & description \\
\hline
\verb'GrB_GT_'$R$     & $R \times R \rightarrow $\verb'bool' & $z = (x >   y)$ & greater than \\
\verb'GrB_LT_'$R$     & $R \times R \rightarrow $\verb'bool' & $z = (x <   y)$ & less than  \\
\verb'GrB_GE_'$R$     & $R \times R \rightarrow $\verb'bool' & $z = (x \ge y)$ & greater than or equal \\
\verb'GrB_LE_'$R$     & $R \times R \rightarrow $\verb'bool' & $z = (x \le y)$ & less than or equal  \\
\hline
\end{tabular}
}
\vspace{0.2in}

GraphBLAS has four built-in binary operators that operate purely in
the Boolean domain.  The first three are identical to the \verb'GxB_L*_BOOL'
operators described above, just with a shorter name.  The \verb'GrB_LXNOR'
operator is the same as \verb'GrB_EQ_BOOL'.

\vspace{0.2in}
{\footnotesize
\begin{tabular}{|llll|}
\hline
\multicolumn{4}{|c|}{Binary operators for the boolean type only} \\
\hline
GraphBLAS name        & types (domains)            & $z=f(x,y)$      & description \\
\hline
% 3 bool x bool -> bool
\verb'GrB_LOR'        & \verb'bool'
                        $\times$ \verb'bool'
                        $\rightarrow$ \verb'bool'  & $z = x \vee    y $ & logical OR \\
\verb'GrB_LAND'       & \verb'bool'
                        $\times$ \verb'bool'
                        $\rightarrow$ \verb'bool'  & $z = x \wedge  y $ & logical AND \\
\verb'GrB_LXOR'       & \verb'bool'
                        $\times$ \verb'bool'
                        $\rightarrow$ \verb'bool'  & $z = x \veebar y $ & logical XOR \\
\verb'GrB_LXNOR'      & \verb'bool'
                        $\times$ \verb'bool'
                        $\rightarrow$ \verb'bool'  & $z = \lnot (x \veebar y) $ & logical XNOR \\
\hline
\end{tabular}
}
\vspace{0.2in}

The following operators are defined for real floating-point types only (\verb'GrB_FP32' and  \verb'GrB_FP64').
They are identical to the C11 functions of the same name.  The last one in the table constructs
the corresponding complex type.

\vspace{0.2in}
{\footnotesize
\begin{tabular}{|llll|}
\hline
\multicolumn{4}{|c|}{Binary operators for the real floating-point types only} \\
\hline
GraphBLAS name        & types (domains)            & $z=f(x,y)$      & description \\
\hline
\verb'GxB_ATAN2_'$F$     & $F \times F \rightarrow F$ & $z = \tan^{-1}(y/x)$ & 4-quadrant arc tangent  \\
\verb'GxB_HYPOT_'$F$     & $F \times F \rightarrow F$ & $z = \sqrt{x^2+y^2}$ & hypotenuse \\
\verb'GxB_FMOD_'$F$      & $F \times F \rightarrow F$ &                      & C11 \verb'fmod' \\
\verb'GxB_REMAINDER_'$F$ & $F \times F \rightarrow F$ &                      & C11 \verb'remainder' \\
\verb'GxB_LDEXP_'$F$     & $F \times F \rightarrow F$ &                      & C11 \verb'ldexp' \\
\verb'GxB_COPYSIGN_'$F$  & $F \times F \rightarrow F$ &                      & C11 \verb'copysign' \\
\hline
\verb'GxB_CMPLX_'$F$     & $F \times F \rightarrow Z$ & $z = x + y \times i$ & complex from real \& imag \\
\hline
\end{tabular}
}
\vspace{0.2in}

Eight bitwise operators are predefined for signed and unsigned integers.

\vspace{0.2in}
{\footnotesize
\begin{tabular}{|llll|}
\hline
\multicolumn{4}{|c|}{Binary operators for signed and unsigned integers} \\
\hline
GraphBLAS name        & types (domains)            & $z=f(x,y)$      & description \\
\hline
\verb'GrB_BOR_'$I$    & $I \times I \rightarrow I$ & \verb'z=x|y'    & bitwise logical OR \\
\verb'GrB_BAND_'$I$   & $I \times I \rightarrow I$ & \verb'z=x&y'    & bitwise logical AND \\
\verb'GrB_BXOR_'$I$   & $I \times I \rightarrow I$ & \verb'z=x^y'    & bitwise logical XOR \\
\verb'GrB_BXNOR_'$I$  & $I \times I \rightarrow I$ & \verb'z=~(x^y)' & bitwise logical XNOR \\
\hline
\verb'GxB_BGET_'$I$    & $I \times I \rightarrow I$  & & get bit y of x \\
\verb'GxB_BSET_'$I$    & $I \times I \rightarrow I$  & & set bit y of x \\
\verb'GxB_BCLR_'$I$    & $I \times I \rightarrow I$  & & clear bit y of x \\
\verb'GxB_BSHIFT_'$I$  & $I \times $\verb'int8'$  \rightarrow I$ & & bit shift \\
\hline
\end{tabular}
}
\vspace{0.2in}

There are two sets of built-in comparators in SuiteSparse:Graph\-BLAS,
but they are not redundant.  They are identical except for the type (domain) of
their output, $z$.  The \verb'GrB_EQ_'$T$ and related operators compare their
inputs of type $T$ and produce a Boolean result of true or false.  The
\verb'GxB_ISEQ_'$T$ and related operators compute the same thing and produce a
result with same type $T$ as their input operands, returning one for true or
zero for false.  The \verb'IS*' comparators are useful when combining
comparators with other non-Boolean operators.  For example, a \verb'PLUS-ISEQ'
semiring counts how many terms are true.  With this semiring,
matrix multiplication ${\bf C=AB}$ for two weighted undirected graphs ${\bf A}$
and ${\bf B}$ computes $c_{ij}$ as the number of edges node $i$ and $j$ have in
common that have identical edge weights.  Since the output type of the
``multiplier'' operator in a semiring must match the type of its monoid, the
Boolean \verb'EQ' cannot be combined with a non-Boolean \verb'PLUS' monoid to
perform this operation.

Likewise, SuiteSparse:GraphBLAS has two sets of logical OR, AND, and XOR
operators.  Without the \verb'_'$T$ suffix, the three operators \verb'GrB_LOR',
\verb'GrB_LAND', and \verb'GrB_LXOR' operate purely in the Boolean domain,
where all input and output types are \verb'GrB_BOOL'.  The second set
(\verb'GxB_LOR_'$T$ \verb'GxB_LAND_'$T$ and \verb'GxB_LXOR_'$T$) provides
Boolean operators to all 11 real domains, implicitly typecasting their inputs from
type $T$ to Boolean and returning a value of type $T$ that is 1 for true or
zero for false.  The set of \verb'GxB_L*_'$T$ operators are useful since they
can be combined with non-Boolean monoids in a semiring.

Floating-point operations follow the IEEE 754 standard.  Thus, computing $x/0$
for a floating-point $x$ results in \verb'+Inf' if $x$ is positive, \verb'-Inf'
if $x$ is negative, and \verb'NaN' if $x$ is zero.  The application is not
terminated.  However, integer division by zero normally terminates an
application.  SuiteSparse:GraphBLAS avoids this by adopting the same rules as
MATLAB, which are analogous to how the IEEE standard handles floating-point
division by zero.  For integers, when $x$ is positive, $x/0$ is the largest
positive integer, for negative $x$ it is the minimum integer, and 0/0 results
in zero.  For example, for an integer $x$ of type \verb'GrB_INT32', 1/0 is
$2^{31}-1$ and (-1)/0 is $-2^{31}$.  Refer to Section~\ref{type} for a list of
integer ranges.

%===============================================================================
\subsubsection{GraphBLAS binary operators based on index binary operators}
%===============================================================================

Eight binary operators based on underlying index binary operators are
predefined.  They differ when used in a semiring and when used in
\verb'GrB_eWise*' and \verb'GrB_apply'.  These index-based binary operators
cannot be used in \verb'GrB_build', nor can they be used as the \verb'accum'
operator for any operation.

The built-in index-based binary operators do not depend on the type or numerical value
of their inputs, just their position in a matrix or vector.  For a vector, $j$
is always 0, and $i$ is the index into the vector.  There are two types $N$
available: \verb'INT32' and \verb'INT64', which is the type of the output $z$.
User-defined index-based operators are not defined by \verb'GrB_BinaryOp_new',
but by \verb'GxB_BinaryOp_new_IndexOp' instead.  See Section~\ref{idxbinaryop}
for details.

\vspace{0.2in}
{\footnotesize
\begin{tabular}{|llll|}
\hline
\multicolumn{4}{|c|}{Index-based binary operators for any type (including user-defined)} \\
\multicolumn{4}{|c|}{when used as a multiplicative operator in a semiring} \\
\hline
GraphBLAS name            & types (domains)   & $z=f(a_{ik},b_{kj})$      & description \\
\hline
\verb'GxB_FIRSTI_'$N$    & $ \rightarrow N$  & $z = i$       & row index of $a_{ik}$ (0-based) \\
\verb'GxB_FIRSTI1_'$N$   & $ \rightarrow N$  & $z = i+1$     & row index of $a_{ik}$ (1-based) \\
\verb'GxB_FIRSTJ_'$N$    & $ \rightarrow N$  & $z = k$       & column index of $a_{ik}$ (0-based) \\
\verb'GxB_FIRSTJ1_'$N$   & $ \rightarrow N$  & $z = k+1$     & column index of $a_{ik}$ (1-based) \\
\verb'GxB_SECONDI_'$N$   & $ \rightarrow N$  & $z = k$       & row index of $b_{kj}$ (0-based) \\
\verb'GxB_SECONDI1_'$N$  & $ \rightarrow N$  & $z = k+1$     & row index of $b_{kj}$ (1-based) \\
\verb'GxB_SECONDJ_'$N$   & $ \rightarrow N$  & $z = j$       & column index of $b_{kj}$ (0-based) \\
\verb'GxB_SECONDJ1_'$N$  & $ \rightarrow N$  & $z = j+1$     & column index of $b_{kj}$ (1-based) \\
\hline
\end{tabular}
}

\vspace{0.2in}
{\footnotesize
\begin{tabular}{|llll|}
\hline
\multicolumn{4}{|c|}{Index-based binary operators for any type (including user-defined)} \\
\multicolumn{4}{|c|}{when used in all other methods} \\
\hline
GraphBLAS name            & types (domains)   & $z=f(a_{ij},b_{ij})$      & description \\
\hline
\verb'GxB_FIRSTI_'$N$    & $ \rightarrow N$  & $z = i$       & row index of $a_{ij}$ (0-based) \\
\verb'GxB_FIRSTI1_'$N$   & $ \rightarrow N$  & $z = i+1$     & row index of $a_{ij}$ (1-based) \\
\verb'GxB_FIRSTJ_'$N$    & $ \rightarrow N$  & $z = j$       & column index of $a_{ij}$ (0-based) \\
\verb'GxB_FIRSTJ1_'$N$   & $ \rightarrow N$  & $z = j+1$     & column index of $a_{ij}$ (1-based) \\
\verb'GxB_SECONDI_'$N$   & $ \rightarrow N$  & $z = i$       & row index of $b_{ij}$ (0-based) \\
\verb'GxB_SECONDI1_'$N$  & $ \rightarrow N$  & $z = i+1$     & row index of $b_{ij}$ (1-based) \\
\verb'GxB_SECONDJ_'$N$   & $ \rightarrow N$  & $z = j$       & column index of $b_{ij}$ (0-based) \\
\verb'GxB_SECONDJ1_'$N$  & $ \rightarrow N$  & $z = j+1$     & column index of $b_{ij}$ (1-based) \\
\hline
\end{tabular}
}
\vspace{0.2in}

Finally, one special binary operator can only be used as input to
\verb'GrB_Matrix_build' or \verb'GrB_Vector_build': the \verb'GxB_IGNORE_DUP'
operator.  If \verb'dup' is \verb'NULL', any duplicates in the \verb'GrB*build'
methods result in an error.  If \verb'dup' is the special binary operator
\verb'GxB_IGNORE_DUP', then any duplicates are ignored.  If duplicates appear,
the last one in the list of tuples is taken and the prior ones ignored.  This
is not an error.

The next sections define the following methods for the \verb'GrB_BinaryOp'
object:

\vspace{0.2in}
{\footnotesize
\begin{tabular}{lll}
\hline
GraphBLAS function   & purpose                                      & Section \\
\hline
\verb'GrB_BinaryOp_new'   & create a user-defined binary operator   & \ref{binaryop_new} \\
\verb'GxB_BinaryOp_new'   & create a named user-defined binary operator   & \ref{binaryop_new_named} \\
\verb'GrB_BinaryOp_wait'  & wait for a user-defined binary operator & \ref{binaryop_wait} \\
\verb'GrB_BinaryOp_free'  & free a user-defined binary operator     & \ref{binaryop_free} \\
\verb'GrB_get'           & get properties of an operator    & \ref{get_set_binop} \\
\verb'GrB_set'           & set the operator name/definition & \ref{get_set_binop} \\
\hline
\end{tabular}
}
\vspace{0.2in}

\newpage
%-------------------------------------------------------------------------------
\subsubsection{{\sf GrB\_BinaryOp\_new:} create a user-defined binary operator}
%-------------------------------------------------------------------------------
\label{binaryop_new}

\begin{mdframed}[userdefinedwidth=6in]
{\footnotesize
\begin{verbatim}
GrB_Info GrB_BinaryOp_new
(
    GrB_BinaryOp *binaryop,         // handle for the new binary operator
    void *function,                 // pointer to the binary function
    GrB_Type ztype,                 // type of output z
    GrB_Type xtype,                 // type of input x
    GrB_Type ytype                  // type of input y
) ;
\end{verbatim}
}\end{mdframed}

\verb'GrB_BinaryOp_new' creates a new binary operator.  The new operator is
returned in the \verb'binaryop' handle, which must not be \verb'NULL' on input.
On output, its contents contains a pointer to the new binary operator.

The three types \verb'xtype', \verb'ytype', and \verb'ztype' are the GraphBLAS
types of the inputs $x$ and $y$, and output $z$ of the user-defined function
$z=f(x,y)$.  These types may be built-in types or user-defined types, in any
combination.  The three types need not be the same, but they must be previously
defined before passing them to \verb'GrB_BinaryOp_new'.

The final argument to \verb'GrB_BinaryOp_new' is a pointer to a user-defined
function with the following signature:

    {\footnotesize
    \begin{verbatim}
    void (*f) (void *z, const void *x, const void *y) ; \end{verbatim} }

When the function \verb'f' is called, the arguments \verb'z', \verb'x', and
\verb'y' are passed as \verb'(void *)' pointers, but they will be pointers to
values of the correct type, defined by \verb'ztype', \verb'xtype', and
\verb'ytype', respectively, when the operator was created.

{\bf NOTE:} SuiteSparse:GraphBLAS may call the function with the pointers
\verb'z' and \verb'x' equal to one another, in which case \verb'z=f(z,y)'
should be computed.  Future versions may use additional pointer aliasing.

\newpage
%-------------------------------------------------------------------------------
\subsubsection{{\sf GxB\_BinaryOp\_new:} create a named user-defined binary operator}
%-------------------------------------------------------------------------------
\label{binaryop_new_named}

\begin{mdframed}[userdefinedwidth=6in]
{\footnotesize
\begin{verbatim}
GrB_Info GxB_BinaryOp_new
(
    GrB_BinaryOp *op,               // handle for the new binary operator
    GxB_binary_function function,   // pointer to the binary function
    GrB_Type ztype,                 // type of output z
    GrB_Type xtype,                 // type of input x
    GrB_Type ytype,                 // type of input y
    const char *binop_name,         // name of the user function
    const char *binop_defn          // definition of the user function
) ;
\end{verbatim} }\end{mdframed}

Creates a named \verb'GrB_BinaryOp'.  Only the first 127 characters of
\verb'binop_name' are used.  The \verb'binop_defn' is a string containing the
entire function itself.  For example:

{\footnotesize
\begin{verbatim}
void absdiff (double *z, double *x, double *y) { (*z) = fabs ((*x) - (*y)) ; } ;
...
GrB_Type AbsDiff ;
GxB_BinaryOp_new (&AbsDiff, absdiff, GrB_FP64, GrB_FP64, GrB_FP64, "absdiff",
  "void absdiff (double *z, double *x, double *y) { (*z) = fabs ((*x) - (*y)) ; }") ; \end{verbatim}}

The two strings \verb'binop_name' and \verb'binop_defn' are optional, but are
required to enable the JIT compilation of kernels that use this operator.

If the JIT is enabled, or if the corresponding JIT kernel has been copied
into the \verb'PreJIT' folder, the \verb'function' may be \verb'NULL'.  In this
case, a JIT kernel is compiled that contains just the user-defined function.
If the JIT is disabled and the \verb'function' is \verb'NULL', this method
returns \verb'GrB_NULL_POINTER'.

The above example is identical to the following usage, except that
\verb'GrB_BinaryOp_new' requires a non-NULL function pointer.

{\footnotesize
\begin{verbatim}
void absdiff (double *z, double *x, double *y) { (*z) = fabs ((*x) - (*y)) ; } ;
...
GrB_Type AbsDiff ;
GrB_BinaryOp_new (&AbsDiff, absdiff, GrB_FP64, GrB_FP64, GrB_FP64) ;
GrB_set (AbsDiff, "absdiff", GxB_JIT_C_NAME) ;
GrB_set (AbsDiff,
  "void absdiff (double *z, double *x, double *y) { (*z) = fabs ((*x) - (*y)) ; }",
  GxB_JIT_C_DEFINITION) ;\end{verbatim}}

%-------------------------------------------------------------------------------
\subsubsection{{\sf GrB\_BinaryOp\_wait:} wait for a binary operator}
%-------------------------------------------------------------------------------
\label{binaryop_wait}

\begin{mdframed}[userdefinedwidth=6in]
{\footnotesize
\begin{verbatim}
GrB_Info GrB_wait               // wait for a user-defined binary operator
(
    GrB_BinaryOp binaryop,      // binary operator to wait for
    int mode                    // GrB_COMPLETE or GrB_MATERIALIZE
) ;
\end{verbatim}
}\end{mdframed}

After creating a user-defined binary operator, a GraphBLAS library may choose
to exploit non-blocking mode to delay its creation.  Currently,
SuiteSparse:GraphBLAS currently does nothing for except to ensure that the
\verb'binaryop' is valid.

%-------------------------------------------------------------------------------
\subsubsection{{\sf GrB\_BinaryOp\_free:} free a user-defined binary operator}
%-------------------------------------------------------------------------------
\label{binaryop_free}

\begin{mdframed}[userdefinedwidth=6in]
{\footnotesize
\begin{verbatim}
GrB_Info GrB_free                   // free a user-created binary operator
(
    GrB_BinaryOp *binaryop          // handle of binary operator to free
) ;
\end{verbatim}
} \end{mdframed}

\verb'GrB_BinaryOp_free' frees a user-defined binary operator.
Either usage:

    {\small
    \begin{verbatim}
    GrB_BinaryOp_free (&op) ;
    GrB_free (&op) ; \end{verbatim}}

\noindent
frees the \verb'op' and sets \verb'op' to \verb'NULL'.
It safely does nothing if passed a \verb'NULL'
handle, or if \verb'op == NULL' on input.
It does nothing at all if passed a built-in binary operator.

%-------------------------------------------------------------------------------
\subsubsection{{\sf ANY} and {\sf PAIR} ({\sf ONEB}) operators}
%-------------------------------------------------------------------------------
\label{any_pair}

The \verb'GxB_PAIR' operator (also called \verb'GrB_ONEB') is simple to describe:
just $f(x,y)=1$.  It is called
the \verb'PAIR' operator since it returns $1$ in a semiring when a pair of
entries $a_{ik}$ and $b_{kj}$ is found in the matrix multiply.  This operator
is simple yet very useful.  It allows purely structural computations to be
performed on matrices of any type, without having to typecast them to Boolean
with all values being true.  Typecasting need not be performed on the inputs to
the \verb'PAIR' operator, and the \verb'PAIR' operator does not need to access
the values of the matrix.  This cuts memory accesses, so it is a very fast
operator to use.

The \verb'GxB_PAIR_T' operator is a SuiteSparse:GraphBLAS extension.
It has since been added to the v2.0 C API Specification as \verb'GrB_ONEB_T'.
They are identical, but the latter name should be used for compatibility
with other GraphBLAS libraries.

The \verb'ANY' operator is very unusual, but very powerful.  It is the function
$f_{\mbox{any}}(x,y)=x$, or $y$, where GraphBLAS has to freedom to select
either $x$, or $y$, at its own discretion.  Do not confuse the \verb'ANY'
operator with the \verb'any' function in MATLAB/Octave, which computes a reduction
using the logical OR operator.

The \verb'ANY' function is associative and commutative, and can thus serve as
an operator for a monoid.  The selection of $x$ are $y$ is not randomized.
Instead, SuiteSparse:GraphBLAS uses this freedom to compute as fast a result as
possible.  When used as the monoid in a dot product, \[ c_{ij} = \sum_k a_{ik}
b_{kj} \] for example, the computation can terminate as soon as any matching
pair of entries is found.  When used in a parallel saxpy-style computation, the
\verb'ANY' operator allows for a relaxed form of synchronization to be used,
resulting in a fast benign race condition.

Because of this benign race condition, the result of the \verb'ANY' monoid can
be non-deterministic, unless it is coupled with the \verb'PAIR' multiplicative
operator.  In this case, the \verb'ANY_PAIR' semiring will return a
deterministic result, since $f_{\mbox{any}}(1,1)$ is always 1.

When paired with a different operator, the results are non-deterministic.  This
gives a powerful method when computing results for which any value selected by
the \verb'ANY' operator is valid.  One such example is the breadth-first-search
tree.  Suppose node $j$ is at level $v$, and there are multiple nodes $i$ at
level $v-1$ for which the edge $(i,j)$ exists in the graph.  Any of these nodes
$i$ can serve as a valid parent in the BFS tree.  Using the \verb'ANY'
operator, GraphBLAS can quickly compute a valid BFS tree; if it used again on
the same inputs, it might return a different, yet still valid, BFS tree, due to
the non-deterministic nature of intra-thread synchronization.

