
\newpage
%%%%%%%%%%%%%%%%%%%%%%%%%%%%%%%%%%%%%%%%%%%%%%%%%%%%%%%%%%%%%%%%%%%%%%%%%%%%%%%%
\section{Iso-Valued Matrices and Vectors } %%%%%%%%%%%%%%%%%%%%%%%%%%%%%%%%%%%%%
%%%%%%%%%%%%%%%%%%%%%%%%%%%%%%%%%%%%%%%%%%%%%%%%%%%%%%%%%%%%%%%%%%%%%%%%%%%%%%%%
\label{iso}

The GraphBLAS C API states that the entries in all \verb'GrB_Matrix' and
\verb'GrB_Vector' objects have a numerical value, with either a built-in or
user-defined type.  Representing an unweighted graph requires a value to be
placed on each edge, typically $a_{ij}=1$.  Adding a structure-only data type
would not mix well with the rest of GraphBLAS, where all operators, monoids,
and semirings need to operate on a value, of some data type.  And yet
unweighted graphs are very important in graph algorithms.

The solution is simple, and exploiting it in SuiteSparse:GraphBLAS requires
nearly no extensions to the GraphBLAS C API.   SuiteSparse:GraphBLAS can often
detect when the user application is creating a matrix or vector where all
entries in the sparsity pattern take on the same numerical value.

For example, ${\bf C \langle C \rangle} = 1$, when the mask is structural, sets
all entries in $\bf C$ to the value 1.  SuiteSparse:GraphBLAS detects this, and
performs this assignment in $O(1)$ time.  It stores a single copy of this
``iso-value'' and sets an internal flag in the opaque data structure for $\bf
C$, which states that all entries in the pattern of $\bf C$ are equal to 1.
This saves both time and memory and allows for the efficient representation of
sparse adjacency matrices of unweighted graphs, yet does not change the C API.
To the user application, it still appears that $\bf C$ has \verb'nvals(C)'
entries, all equal to 1.

Creating and operating on iso-valued matrices (or just {\em iso matrices} for
short) is significantly faster than creating matrices with different data
values.  A matrix that is iso requires only $O(1)$ space for its numerical
values.  The sparse and hypersparse formats require an additional $O(n+e)$ or
$O(e)$ integer space to hold the pattern of an $n$-by-$n$ matrix \verb'C',
respectively, and a matrix \verb'C' in bitmap format requires $O(n^2)$ space
for the bitmap.  A full matrix requires no integer storage, so a matrix that is
both iso and full requires only $O(1)$ space, regardless of its dimension.

The sections below a describe the methods that can be used to create iso
matrices and vectors.  Let $a$, $b$, and $c$ denote the iso values of \verb'A',
\verb'B', and \verb'C', respectively.

%-------------------------------------------------------------------------------
\subsection{Using iso matrices and vectors in a graph algorithm}
%-------------------------------------------------------------------------------
\label{iso_usage}

There are two primary useful ways to use iso-valued matrices and vectors: (1)
as iso sparse/hypersparse adjacency matrices for unweighted graphs, and (2) as
iso full matrices or vectors used with operations that do not need to access
all of the content of the iso full matrix or vector.

In the first use case, simply create a \verb'GrB_Matrix' with values that are
all the same (those in the sparsity pattern).  The
\verb'GxB_Matrix_build_Scalar' method can be used for this, since it
guarantees that the time and work spent on the numerical part of the array
is only $O(1)$.  The method still must spend $O(e)$ or $O(e \log e)$ time
on the integer arrays that represent the sparsity pattern, but the reduction
in time and work on the numerical part of the matrix will improve performance.

The use of \verb'GxB_Matrix_build_Scalar' is optional.  Matrices can also be
constructed with \verb'GrB*' methods.  In particular, \verb'GrB_Matrix_build_*'
can be used.  It first builds a non-iso matrix and then checks if all of the
values are the same, after assembling any duplicate entries.  This does not
save time or memory for the construction of the matrix itself, but it will
lead to savings in time and memory later on, when the matrix is used.

To ensure a matrix \verb'C' is iso-valued, simply use \verb'GrB_assign' to
compute \verb'C<C,struct>=1', or assign whatever value of scalar you wish.
It is essential to use a structural mask.  Otherwise, it is not clear that
all entries in \verb'C' will be assigned the same value.  The following
code takes $O(1)$ time, and it resets the size of the numerical part of the
\verb'C' matrix to be $O(1)$ in size:

{\footnotesize
\begin{verbatim}
    bool scalar = true ;
    GrB_Matrix_assign (C, C, NULL, scalar, GrB_ALL, nrows, GrB_ALL, ncols,
        GrB_DESC_S) ; \end{verbatim}}

The MATLAB/Octave analog of the code above is \verb'C=spones(C)'.

The second case for where iso matrices and vectors are useful is to use them
with operations that do not necessarily access all of their content.
Suppose you have a matrix \verb'A' of arbitrarily large dimension (say
\verb'n'-by-\verb'n' where \verb'n=2^60', of type \verb'GrB_FP64').  A matrix
this large can be represented by SuiteSparse:GraphBLAS, but only in a
hypersparse form.

Now, suppose you wish to compute the maximum value in each row, reducing the
matrix to a vector.  This can be done with \verb'GrB_reduce':

{\footnotesize
\begin{verbatim}
    GrB_Vector_new (&v, GrB_FP64, n) ;
    GrB_reduce (v, NULL, GrB_MAX_MONOID_FP64, A, NULL) ; \end{verbatim}}

It can also be done with \verb'GrB_mxv', by creating an iso full vector
\verb'x'.  The creation of \verb'x' takes $O(1)$ time and memory,
and the \verb'GrB_mxv' computation takes $O(e)$ time (with modest assumptions;
if \verb'A' needs to be transposed the time would be $O(e \log e)$).

{\footnotesize
\begin{verbatim}
    GrB_Vector_new (&v, GrB_FP64, n) ;
    GrB_Vector_new (&x, GrB_FP64, n) ;
    GrB_assign (x, NULL, NULL, 1, GrB_ALL, n, NULL) ;
    GrB_mxv (v, NULL, NULL, GrB_MAX_FIRST_SEMIRING_FP64, A, x, NULL) ; \end{verbatim}}

The above computations are identical in SuiteSparse:GraphBLAS.  Internally,
\verb'GrB_reduce' creates \verb'x' and calls \verb'GrB_mxv'.  Using
\verb'GrB_mxm' directly gives the user application additional flexibility in
creating new computations that exploit the multiplicative operator in the
semiring.  \verb'GrB_reduce' always uses the \verb'FIRST' operator in its
semiring, but any other binary operator can be used instead when using
\verb'GrB_mxv'.

%-------------------------------------------------------------------------------
\subsection{Iso matrices from matrix multiplication}
%-------------------------------------------------------------------------------
\label{iso_mxm}

Consider \verb'GrB_mxm', \verb'GrB_mxv', and \verb'GrB_vxm', and
    let \verb'C=A*B', where no mask is present, or \verb'C<M>=A*B' where
    \verb'C' is initially empty.  If \verb'C' is not initially empty,
    then these rules apply to a temporary matrix \verb'T<M>=A*B', which is
    initially empty and is then assigned to \verb'C' via \verb'C<M>=T'.

    The iso property of \verb'C' is determined with the following rules,
    where the first rule that fits defines the property and value of \verb'C'.

    \begin{itemize}
    \item If the semiring includes a index-based multiplicative operator
    (\verb'GxB_FIRSTI', \verb'GrB_SECONDI', and related operators), then
    \verb'C' is never iso.

    \item Define an {\em iso-monoid} as a built-in monoid with the property
    that reducing a set of $n>1$ identical values $x$ returns the same value
    $x$.  These are the \verb'MIN' \verb'MAX' \verb'LOR' \verb'LAND' \verb'BOR'
    \verb'BAND' and \verb'ANY' monoids.  All other monoids are not iso monoids:
    \verb'PLUS', \verb'TIMES', \verb'LXNOR', \verb'EQ', \verb'BXOR',
    \verb'BXNOR', and all user-defined monoids.   Currently, there is no
    mechanism for telling SuiteSparse:GraphBLAS that a user-defined monoid
    is an iso-monoid.

    \item If the multiplicative op is \verb'PAIR' (same as \verb'ONEB'),
    and the monoid is an
    iso-monoid, or the \verb'EQ' or \verb'TIMES' monoids, then \verb'C' is
    iso with a value of 1.

    \item If both \verb'B' and the monoid are iso, and the multiplicative op is
    \verb'SECOND' or \verb'ANY', then \verb'C' is iso with a value of $b$.

    \item If both \verb'A' and the monoid are iso, and the multiplicative op is
    \verb'FIRST' or \verb'ANY', then \verb'C' is iso with a value of $a$.

    \item If \verb'A', \verb'B', and the monoid are all iso, then \verb'C'
    is iso, with a value $c=f(a,b)$, where $f$ is any multiplicative op
    (including user-defined, which assumes that a user-defined $f$ has no
    side effects).

    \item If \verb'A' and \verb'B' are both iso and full (all entries present,
    regardless of the format of the matrices), then \verb'C' is iso and full.
    Its iso value is computed in $O(\log(n))$ time, via a reduction of $n$
    copies of the value $t=f(a,b)$ to a scalar.  The storage required to
    represent \verb'C' is just $O(1)$, regardless of its dimension.
    Technically, the \verb'PLUS' monoid could be computed as $c=nt$ in $O(1)$
    time, but the log-time reduction works for any monoid, including
    user-defined ones.

    \item Otherwise, \verb'C' is not iso.
    \end{itemize}

%-------------------------------------------------------------------------------
\subsection{Iso matrices from eWiseMult and kronecker}
%-------------------------------------------------------------------------------
\label{iso_emult}

Consider \verb'GrB_eWiseMult'.  Let
\verb'C=A.*B', or \verb'C<M>=A.*B' with any mask and where \verb'C' is
initially empty, where \verb'.*' denotes a binary operator $f(x,y)$
applied with \verb'eWiseMult'.  These rules also apply to \verb'GrB_kronecker'.

    \begin{itemize}
    \item If the operator is index-based (\verb'GxB_FIRSTI' and related) then
    \verb'C' is not iso.

    \item If the op is \verb'PAIR' (same as \verb'ONEB'),
        then \verb'C' is iso with $c=1$.

    \item If \verb'B' is iso and the op is \verb'SECOND' or \verb'ANY',
        then \verb'C' is iso with $c=b$.

    \item If \verb'A' is iso and the op is \verb'FIRST' or \verb'ANY',
        then \verb'C' is iso with $c=a$.

    \item If both \verb'A' and \verb'B' are iso,
        then \verb'C' is iso with $c=f(a,b)$.

    \item Otherwise, \verb'C' is not iso.
    \end{itemize}

%-------------------------------------------------------------------------------
\subsection{Iso matrices from eWiseAdd}
%-------------------------------------------------------------------------------
\label{iso_add}

Consider \verb'GrB_eWiseAdd', and also the accumulator phase of \verb'C<M>+=T'
when an accumulator operator is present.  Let \verb'C=A+B', or \verb'C<M>=A+B'
with any mask and where \verb'C' is initially empty.

    \begin{itemize}
    \item If both \verb'A' and \verb'B' are full (all entries present), then
    the rules for \verb'eWiseMult' in Section~\ref{iso_emult} are used
    instead.

    \item If the operator is index-based (\verb'GxB_FIRSTI' and related) then
    \verb'C' is not iso.

    \item If $a$ and $b$ differ (when typecasted to the type of \verb'C'),
    then \verb'C' is not iso.

    \item If $c=f(a,b) = a = b$ holds, then \verb'C' is iso,
    where $f(a,b)$ is the operator.

    \item Otherwise, \verb'C' is not iso.
    \end{itemize}

%-------------------------------------------------------------------------------
\subsection{Iso matrices from eWiseUnion}
%-------------------------------------------------------------------------------
\label{iso_union}

\verb'GxB_eWiseUnion' is very similar to \verb'GrB_eWiseAdd', but the rules
for when the result is iso-valued are very different.

    \begin{itemize}
    \item If both \verb'A' and \verb'B' are full (all entries present), then
    the rules for \verb'eWiseMult' in Section~\ref{iso_emult} are used
    instead.

    \item If the operator is index-based (\verb'GxB_FIRSTI' and related) then
    \verb'C' is not iso.

    \item If the op is \verb'PAIR' (same as \verb'ONEB'),
        then \verb'C' is iso with $c=1$.

    \item If \verb'B' is iso and the op is \verb'SECOND' or \verb'ANY',
        and the input scalar \verb'beta' matches $b$
        (the iso-value of \verb'B'),
        then \verb'C' is iso with $c=b$.

    \item If \verb'A' is iso and the op is \verb'FIRST' or \verb'ANY',
        and the input scalar \verb'alpha' matches $a$
        (the iso-value of \verb'A'),
        then \verb'C' is iso with $c=a$.

    \item If both \verb'A' and \verb'B' are iso,
        and $f(a,b) = f(\alpha,b) = f(a,\beta)$,
        then \verb'C' is iso with $c=f(a,b)$.

    \item Otherwise, \verb'C' is not iso.
    \end{itemize}

%-------------------------------------------------------------------------------
\subsection{Reducing iso matrices to a scalar or vector}
%-------------------------------------------------------------------------------
\label{iso_reduce}

If \verb'A' is iso with $e$ entries, reducing it to a scalar takes $O(\log(e))$
time, regardless of the monoid used to reduce the matrix to a scalar.  Reducing
\verb'A' to a vector \verb'c' is the same as the matrix-vector multiply
\verb"c=A*x" or \verb"c=A'*x", depending on the descriptor, where \verb'x'
is an iso full vector (refer to Section~\ref{iso_mxm}).

%-------------------------------------------------------------------------------
\subsection{Iso matrices from apply}
%-------------------------------------------------------------------------------
\label{iso_apply}

Let \verb'C=f(A)' denote the application of a unary operator \verb'f',
and let \verb'C=f(A,s)' and \verb'C=f(s,A)' denote the application of a binary
operator with \verb's' a scalar.

    \begin{itemize}
    \item If the operator is index-based (\verb'GxB_POSITION*',
    \verb'GxB_FIRSTI', and related) then \verb'C' is not iso.

    \item If the operator is \verb'ONE' or \verb'PAIR' (same as \verb'ONEB'),
        then \verb'C' iso with $c=1$.

    \item If the operator is \verb'FIRST' or \verb'ANY' with \verb'C=f(s,A)',
        then \verb'C' iso with $c=s$.

    \item If the operator is \verb'SECOND' or \verb'ANY' with \verb'C=f(A,s)',
        then \verb'C' iso with $c=s$.

    \item If \verb'A' is iso then \verb'C' is iso, with the following value
        of $c$:

        \begin{itemize}
        \item If the op is \verb'IDENTITY', then $c=a$.
        \item If the op is unary with \verb'C=f(A)', then $c=f(a)$.
        \item If the op is binary with \verb'C=f(s,A)', then $c=f(s,a)$.
        \item If the op is binary with \verb'C=f(A,s)', then $c=f(a,s)$.
        \end{itemize}


    \item Otherwise, \verb'C' is not iso.
    \end{itemize}

%-------------------------------------------------------------------------------
\subsection{Iso matrices from select}
%-------------------------------------------------------------------------------
\label{iso_select}

Let \verb'C=select(A)' denote the application of a \verb'GrB_IndexUnaryOp' operator
in \verb'GrB_select'.

    \begin{itemize}
    \item If \verb'A' is iso, then \verb'C' is iso with $c=a$.
    \item If the operator is any \verb'GrB_VALUE*_BOOL' operator,
        with no typecasting, and the test is true only for a single boolean
        value, then \verb'C' is iso.
    \item If the operator is \verb'GrB_VALUEEQ_*', with no typecasting,
        then \verb'C' is iso, with $c=t$ where $t$ is the value of the scalar
        \verb'y'.
    \item If the operator is \verb'GrB_VALUELE_UINT*', with no typecasting,
        and the scalar \verb'y' is zero, then \verb'C' is iso with $c=0$.
    \item Otherwise, \verb'C' is not iso.
    \end{itemize}

%-------------------------------------------------------------------------------
\subsection{Iso matrices from assign and subassign}
%-------------------------------------------------------------------------------
\label{iso_assign}

These rules are somewhat complex.  Consider the assignment \verb'C<M>(I,J)=...'
with \verb'GrB_assign'.  Internally, this assignment is converted into
\verb'C(I,J)<M(I,J)>=...' and then \verb'GxB_subassign' is used.  Thus,
all of the rules below assume the form \verb'C(I,J)<M>=...' where \verb'M'
has the same size as the submatrix \verb'C(I,J)'.

\subsubsection{Assignment with no accumulator operator}

If no accumulator operator is present, the following rules are used.

\begin{itemize}
\item
For matrix assignment, \verb'A' must be iso.  For scalar assignment, the single
scalar is implicitly expanded into an iso matrix \verb'A' of the right size.
If these rules do not hold, \verb'C' is not iso.

\item
If \verb'A' is not iso, or if \verb'C' is not iso on input, then \verb'C' is
not iso on output.

\item
If \verb'C' is iso or empty on input, and \verb'A' is iso (or scalar assignment
is begin performed) and the iso values $c$ and $a$ (or the scalar $s$) match,
then the following forms of assignment result in an iso matrix \verb'C'  on
output:

                \begin{itemize}
                \item \verb'C(I,J) = scalar'
                \item \verb'C(I,J)<M> = scalar'
                \item \verb'C(I,J)<!M> = scalar'
                \item \verb'C(I,J)<M,replace> = scalar'
                \item \verb'C(I,J)<!M,replace> = scalar'
                \item \verb'C(I,J) = A'
                \item \verb'C(I,J)<M> = A'
                \item \verb'C(I,J)<!M> = A'
                \item \verb'C(I,J)<M,replace> = A'
                \item \verb'C(I,J)<!M,replace> = A'
                \end{itemize}

\item
For these forms of assignment, \verb'C' is always iso on output, regardless
of its iso property on input:

                \begin{itemize}
                \item \verb'C = scalar'
                \item \verb'C<M,struct>=scalar'; C empty on input.
                \item \verb'C<C,struct>=scalar'
                \end{itemize}

\item
For these forms of assignment, \verb'C' is always iso on output if \verb'A'
is iso:

                \begin{itemize}
                \item \verb'C = A'
                \item \verb'C<M,str> = A'; C empty on input.
                \end{itemize}
\end{itemize}


\subsubsection{Assignment with an accumulator operator}

If an accumulator operator is present, the following rules are used.
Index-based operators (\verb'GxB_FIRSTI' and related) cannot be used as
accumulator operators, so these rules do not consider that case.

\begin{itemize}
\item
For matrix assignment, \verb'A' must be iso.  For scalar assignment, the single
scalar is implicitly expanded into an iso matrix \verb'A' of the right size.
If these rules do not hold, \verb'C' is not iso.

\item For these forms of assignment \verb'C' is iso if \verb'C' is
empty on input, or if $c=c+a$ for the where $a$ is the iso value of \verb'A' or
the value of the scalar for scalar assignment.

                \begin{itemize}
                \item \verb'C(I,J) += scalar'
                \item \verb'C(I,J)<M> += scalar'
                \item \verb'C(I,J)<!M> += scalar'
                \item \verb'C(I,J)<M,replace> += scalar'
                \item \verb'C(I,J)<!M,replace> += scalar'
                \item \verb'C(I,J)<M,replace> += A'
                \item \verb'C(I,J)<!M,replace> += A'
                \item \verb'C(I,J) += A'
                \item \verb'C(I,J)<M> += A'
                \item \verb'C(I,J)<!M> += A '
                \item \verb'C += A'
                \end{itemize}
\end{itemize}

%-------------------------------------------------------------------------------
\subsection{Iso matrices from build methods}
%-------------------------------------------------------------------------------
\label{iso_build}

\verb'GxB_Matrix_build_Scalar' and \verb'GxB_Vector_build_Scalar'
always construct an iso matrix/vector.

\verb'GrB_Matrix_build' and \verb'GrB_Vector_build' can also construct iso
matrices and vectors.  A non-iso matrix/vector is constructed first, and then
the entries are checked to see if they are all equal.  The resulting iso-valued
matrix/vector will be efficient to use and will use less memory than a non-iso
matrix/vector.  However, constructing an iso matrix/vector with
\verb'GrB_Matrix_build' and \verb'GrB_Vector_build' will take more time
and memory than constructing the matrix/vector with
\verb'GxB_Matrix_build_Scalar' or \verb'GxB_Vector_build_Scalar'.

%-------------------------------------------------------------------------------
\subsection{Iso matrices from other methods}
%-------------------------------------------------------------------------------
\label{iso_other}

\begin{itemize}
\item
For \verb'GrB_Matrix_dup' and \verb'GrB_Vector_dup', the output matrix/vector
has the same iso property as the input matrix/vector.

\item
\verb'GrB_*_setElement_*' preserves the iso property of the matrix/vector it
modifies, if the input scalar is equal to the iso value of the matrix/vector.
If the matrix or vector has no entries, the first call to \verb'setElement'
makes it iso.  This allows a sequence of \verb'setElement' calls with the same
scalar value to create an entire iso matrix or vector, if starting from
an empty matrix or vector.

\item
\verb'GxB_Matrix_concat' constructs an iso matrix as its result if all input
tiles are either empty or iso.

\item
\verb'GxB_Matrix_split' constructs its output tiles as iso if its input
matrix is iso.

\item
\verb'GxB_Matrix_diag' and \verb'GrB_Matrix_diag' construct an iso matrix if
its input vector is iso.

\item
\verb'GxB_Vector_diag' constructs an iso vector if its input matrix is iso.

\item
\verb'GrB_*extract' constructs an iso matrix/vector if its input matrix/vector
is iso.

\item
\verb'GrB_transpose' constructs an iso matrix if its input is iso.

\item
The \verb'GxB_Container' methods preserve the iso property
of their matrices/vectors.
\end{itemize}

%-------------------------------------------------------------------------------
\subsection{Iso matrices not exploited}
%-------------------------------------------------------------------------------

There are many cases where an matrix may have the iso property but it is not
detected by SuiteSparse:GraphBLAS.  For example, if \verb'A' is non-iso,
\verb'C=A(I,J)' from \verb'GrB_extract' may be iso, if all entries in the
extracted submatrix have the same value.  Future versions of
SuiteSparse:GraphBLAS may extend the rules described in this section to detect
these cases.

